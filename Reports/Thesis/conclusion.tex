\chapter{Conclusion}
\label{cha:conclusion}

This thesis started with explaining the new research area of Electronic Health Record Analytics. We explored the possible impact of this area as it allows to find medical patterns on a large scale. Those patterns range from drug discovery, disease progression for individuals, and reducing medical costs. At the moment several research groups are working on utilizing EHRs to find medical patterns using several methods like querying, statistics, data mining, and artificial intelligence approaches. \\
Our research sought to explore the usage of new machine learning approaches to find correlations between different diagnoses. The correlations found can be used in combination with prediction or classification methods. \\

We introduced a new way on how to apply Word2Vec methods by explaining the link between sentences of words and sequence of medical records. We call this approach a generalized Word2Vec approach and it can be applied on medical data to the correlations between different diagnoses. \\
To make sure the generalized Word2Vec methods can be applied to large-scale medical data, we applied the generalization concept also on DeepWalk. DeepWalk makes it possible to generate a smaller dataset from the original dataset and then apply a Word2Vec approach on this smaller dataset. \\
Besides the exploration on generalizing Word2Vec approaches, we also improved Word2Vec by tackling on of its shortcomings. This shortcoming of Word2Vec is that it is unable to handle unseen instances once it has built his lookup table. We combine a k-nearest neighbors method with Word2Vec and make an estimation of the correlation to other diagnoses for the unseen instance. \\

EXPERIMENTS + Limitations of the experiment

CONCLUSION

For more information about another possible experiment which also combines the Word2Vec approaches with prediction or classification methods, we refer to the next chapter. In this chapter we also talk about some possible improvements and future directions to explore.

%%% Local Variables: 
%%% mode: latex
%%% TeX-master: "thesis"
%%% End: 
