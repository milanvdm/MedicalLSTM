\graphicspath{ {Context/Images/} }


\chapter{Context}
\label{cha:context}

\section{Introduction}

In this chapter we explain the context in which we will be working for this thesis. \\

In section \ref{sec:ehr} we explain what electronic health records are and how these are represented. In section \ref{sec:ehra}, we go further on how the electronic health records can be used to retrieve useful medical information. We explain different approaches on how to retrieve this information.


\section{Electronic Health Records}
\label{sec:ehr}

An electronic health record (EHR) is a collection of time-stamped data about a patient over a period point of time. It is stored digitally and thus can be established for a large number of patients over a long time period. \\
The data stored in an EHR provides an overview of the patients health information. Health information like demographics, medical history, diagnoses, medications, and such, are stored \cite{HealthIT:online}. \\

Large countries like the US and the UK, are each investing more than $20$ billion dollars into EHR systems \cite{EHRworld:article}. Those systems are adopted by around $70$\% of the physicians. Which means a large number of physicians are using other methods or systems. Also, each country develops his own system which results in a good nationwide coverage but introduces different system around the world. We focus on disease codes in the next section and introduce two standards: one used by mainly insurance companies and the other used by pharmaceutical companies.


\subsection{Disease Codes}

To make EHRs practical it is important to adhere to standards for data formatting. A well documented standard makes it easy to store and extract information from large-scale databases of EHRs. Without the possibility of extracting information, an EHR becomes a simple digital version of medical records on paper. \\
A part of an EHR consists of the diagnosis of the patient. It provides information about his disease trajectory and allows analysis on his health situation. With a uniform system for classifying diseases, it is possible to provide a general picture on health situations of populations.

\subsubsection{ICD-10}

The International Statistical Classification of Diseases and Related Health Problems (ICD) is a medical classification list made by the World Health Organization (WHO) \cite{WHO_ICD:online}. The ICD-10 contains more than $14,400$ codes about diseases, disorders, injuries, and other related health conditions. For example, the code for a sprained ankle is $S93.4$. It also provides hierarchical categories for those codes to allow a more general overview of diseases. ICD is mainly used by insurance companies.


\subsubsection{MedDRA}

The Medical Dictionary for Regulatory Activities (MedDRA) provides medical terminology in the form of disease codes \cite{MedDRA:online}. A MedDRA code is an eight digit numberic code where new terms are assigned sequentially. It does not provide clear hierarchical categories like ICD which can't be understand without a medical background. MedDRA is mainly used by pharmaceutical companies.

\section{EHR Analytics}
\label{sec:ehra}

EHRs provide a massive amount of data which could be used to create useful insights. The data contains the medical history of a patient including medical measurements, diagnosis, prescribed drugs, and demographics. Based on those values, we could obtain the following insights:

\begin{itemize}

\item Effects of drugs
\item Medical costs for certain diseases
\item Duration and recovery percentage of certain diseases
\item Correlation between demographics and certain diseases
\item Link between current health state and health history
\item Prediction of future health states based on history

\end{itemize}

Those insights can be offered on an individual level, which means a right intervention to the right patient at the right time. EHR analytics can be used to have a personalized care and benefits the healthcare system by cutting costs and improve outcomes. \\

In the following sections we talk about current EHR analytic methods.


\subsection{Querying}

Analytics in epidemiology on EHRs is typically done through querying a database \cite{EHRquery:journal}. A specialist can have a certain idea about correlations between conditions or patients. He can support this idea by finding cases in EHRs and analyzing the results of his query. \\
This method is based on the knowledge and experience of a specialist. The information has to be actively sought after and unexpected or complex correlations are not considered. Some complex relations cannot be found because of the limitations of the querying language. A query language is equivalent to first-order logic. Which means non-linear relations in the data cannot be found.

\subsection{Big Data Analytics}

More advanced methods are applied on EHR than querying. In general, they try to find patterns in the EHR data which then can be used to predict outcomes of treatments \cite{EHRbigdata:slides}. \\

Several predictive methods from machine learning can be used and show promising results \cite{EHRmining:article}. Those results are achieved by using non-optimized methods which are applied on the EHR data. We also note that methods used as Multi-layer Perceptron networks are not ideal for prediction of time-series, see section \ref{sec:PatientClassification}. We conclude that there is still a lot of room for improvement. \\
 
More specialized approaches are also applied on EHR data \cite{EHRrecommender:article}. 
An EHR of a patient can be transformed into a matrix structure, see figure \ref{fig:matrixPatient}. On these matrix structures, large-scale data mining algorithms can be applied. Those make it possible to mine temporal patterns in EHR data. The found patterns can be used for prediction later on.

\begin{figure}[H]
	\centering
	\includegraphics[width=\textwidth]{matrixPatient.png}
	\caption{Example of an EHR transformed into a matrix structure \cite{EHRrecommender:article}}
	\label{fig:matrixPatient}
\end{figure}

It is also possible to define patient similarities \cite{EHRsimilarity:article}. So when a patient is similar to a previous patient, his treatment can be based on previous experiences. This is a similar approach to recommender systems.


\subsection{Statistical Analysis}

A more statistical approach is used to find patterns in EHR data on a dataset of the Danish population \cite{Brunak:article}. \\

First we describe the dataset. The dataset which is used to apply the statistical analysis on, are EHRs collected over $15$ years on over $6$ million patients in Denmark. The large size of this dataset makes it possible to retrieve significant results. \\

We start with finding pairs of diagnoses which have a strong correlation between the diagnoses. After finding the correlated pairs, a test for directionality is applied. From this, only the pairs with a high enough correlation for a direction are kept. \\

The directed pairs are then connected when they have overlapping diagnoses into longer trajectories. The trajectories are then clustered. From the clusters, diagnoses can be found which are key in the disease progression. Those key diagnoses could be used to predict disease progression of patients. \\

The found clusters will be used to validate our approach described in chapter \ref{cha:approach}. You can find an example of a clustered trajectory in figure \ref{fig:clusterGraphDanish}.

\begin{figure}[H]
	\centering
	\includegraphics[width=\textwidth]{clusterGraphDanish.png}
	\caption{Cerebrovascular disease trajectory cluster for the Danish population \cite{Brunak:article}}
	\label{fig:clusterGraphDanish}
\end{figure}


\section{Conclusion}

We conclude that EHRs contain important information of a patients medical history and current state. When a large amount of data of EHRs is available, empirical results can be found in the form of patterns. Those pattern can be used to predict and improve medical outcomes on a personal level. \\
The methods we describe vary from simple to very complex. But there is still room for improvement, especially in the field of advanced machine learning algorithms. The results of the Danish paper can be used to have a first validation of our approach. \\

In the next chapter we explain the needed background knowledge to understand the other chapters of this thesis.


%%% Local Variables: 
%%% mode: latex
%%% TeX-master: "thesis"
%%% End: 
