\graphicspath{ {Context/Images/} }


\chapter{Context}
\label{cha:context}

\section{Introduction}
In this chapter 

Context: Depends on chosen problem (Extracting knowledge from EHR )

	Problem: info out of EHR
	State of the art:
	
		Explain Disease codes
		DiseaseProgression
		Explain Danish paper, Matrix group America (recommender), Querying


\section{Electronic Health Records}

An electronic health records (EHR) is a collection of data about a patient at a certain point in time. It is stored digitally and thus can be part of a large number of patients over a long time period.

\subsection{Disease Codes}

An important part of EHRs is applying methods which are uniform and well documented. This makes it easy to store and extract data from large-scale databases of EHRs. A part of an EHR consists of the diagnosis of the patient. It provides information about the disease trajectory of the patient and allows analysis on the health situation of a patient. With a uniform system for classifying diseases, it is possible to provide a general picture on health situations of populations.

\subsubsection{ICD-10}

The International Statistical Classification if Diseases and Related Health Problems (ICD) is a medical classification list made by the World Health Organization (WHO). The ICD contains more than $14,400$ codes about diseases, disorders, injuries, and other related health conditions. It also provides hierarchical categories for those codes to allow a more general overview of diseases.


\subsubsection{MedDRA}

The Medical Dictionary for Regulatory Activities (MedDRA) provides medical terminology in the form of disease codes. It doesn't provide hierarchical categories.

\section{EHR Analytics}

EHRs provide a massive amount of data which could be used to create useful insights. Those insights can be offered on an individual level, which means a right intervention to the right patient at the right time. EHR analytics can be used to have a personalized care for patients and benefits the healthcare system by cutting costs and improved outcomes. \\

In the following sections we talk about current EHR analytic methods.


\subsection{Querying}



\subsection{Recommender Systems}

\subsection{Statistical Analysis}

\subsection{Patient Similarity}


http://www.who.int/classifications/icd/en/

http://arxiv.org/pdf/1112.1668v1.pdf

https://en.wikipedia.org/wiki/Electronic\_health\_record#Technical\_features

file:///media/milan/Data/Chrome\%20Downloads/EHRAnalytics_Stateoftheart\%20(1).pdf

http://www.meddra.org/

https://www.siam.org/meetings/sdm13/sun.pdf

Jimeng Sun, Fei Wang, Jianying Hu, Shahram Edabollahi: Supervised patient similarity measure of
heterogeneous patient records. SIGKDD Explorations 14(1): 16-24 (2012)


\section{Conclusion}
The final section of the chapter gives an overview of the important results
of this chapter. This implies that the introductory chapter and the
concluding chapter don't need a conclusion.



%%% Local Variables: 
%%% mode: latex
%%% TeX-master: "thesis"
%%% End: 
