\chapter{Introduction}
\label{cha:introduction}

In todays medical world, more and more data is stored using Electronic Health Records. Each time a patient goes to a hospital, doctor or receives lab results, those events are stored in the patient's Electronic Health Record. The medical world and governments are interested in EHRs as they can for example provide new insights into disease trajectories, drug treatments, medical costs, or the link between demographics and certain diseases. \\

Due to the increased usage of EHRs, a new research area emerged, namely the area of Electronic Health Record Analytics. EHR analytics is an active research field as a lot of different problems need to be solved, like different codings, privacy, interpretation, and large amounts of data. At the moment several research groups are working on utilizing EHRs to find medical patterns using several methods like querying, statistics, data mining, and artificial intelligence approaches. \\
The methods we mentioned vary from simple to very complex. But there is still room for improvement, especially in the field of advanced machine learning algorithms. This is the focus point of this thesis: applying advanced machine learning algorithms to find patterns in EHRs. \\

An EHR of a patient can be seen as a time series, namely a sequence of EHR events such as visits to the doctor. We make the analogy between sentences of words and sequences of EHR events. Based on this analogy, we propose novel techniques which are generalizations of the Word2Vec approach. \\
Before we apply these Word2Vec methods, we preprocess our EHRs to categorize our EHR events into more general events. For example, a bruise on your right leg and a bruise on your left leg should both be categorized under a bruises event. \\

We introduce generalized Word2Vec. This generalized Word2Vec makes it possible apply Word2Vec on medical data. \\
To make sure the generalized Word2Vec methods can be applied to large-scale medical data, we apply the generalization concept also on DeepWalk. DeepWalk makes it possible to generate a smaller dataset from the original dataset and then apply a Word2Vec approach on this smaller dataset. \\
Besides the exploration on generalizing Word2Vec approaches, we also improve Word2Vec by tackling one of its shortcomings. This shortcoming of Word2Vec is that it is unable to handle unseen instances once it has built his model. We combine a k-nearest neighbors method with Word2Vec and make an estimation of the correlation to other diagnoses for the unseen instance. \\

We compare the model from our Word2Vec methods applied on the OSIM2 dataset to the results of the currently largest study on Danish EHRs. To make it possible to apply our methods on the OSIM2 dataset and generalize our events, we categorize our diagnoses and make a disease code mapping between two standards namely MedDRA and ICD-10. \\

Within the limitations of our validation method, we conclude that our model does match the Danish results well depending on how the matching is defined. Especially since we use several estimations such as the disease code mapping, different datasets, and categorization. \\

In chapter \ref{cha:context}, we describe the general field of EHR and EHR analytics. We also introduce some state-of-the art methods. In chapter \ref{cha:background}, we provide you with the needed background to understand our generalized Word2Vec approaches and afterwards describe our novel techniques. In chapter \ref{cha:implementation}, we describe how we make a model of the OSIM2 dataset using our proposed techniques and how to validate our model. In chapter \ref{cha:futureWork}, we mention some possible improvements and extensions left to explore.

%%% Local Variables: 
%%% mode: latex
%%% TeX-master: "thesis"
%%% End: 
