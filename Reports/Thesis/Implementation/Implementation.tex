\graphicspath{ {Implementation/Images/} }


\chapter{Validation}
\label{cha:implementation}

\section{Introduction}
In this chapter 

DiseaseMapping (generalization)
OSIM
Clusters
TensorFlow
DL4J


\section{Dataset}

To validate the approaches mentioned in chapter \ref{cha:background}, we used a dataset generated by OSIM2. This dataset is used by OMOP to validate their methods to predict the effects of drug treatments. It contains around $10$ million of hypothetical patients based on Thomson Reuters MarketScan Lab Database (MSLR). MSLR contains administrative claims between 2003 ad 2009 from a privately-insured population. \\

The OSIM2 dataset is contains multiple database tables which are dumped as comma-separated values (csv) files. To make it easier to work with this dataset, we joined the multiple files into one file with on each row an event of a patient containing all relevant information. The relevant information which is kept is: birth year, gender, condition type, condition, time difference since previous diagnosis, and season (summer, fall, winter, spring). Thus, one EHR is $7$ dimensional. \\	

Using our approaches on this dataset, we can compare our results to the found clusters in Anders Boeck Jensen et. \cite{Brunak:article}. Although these found clusters are not a golden standard, it is a first validation point for our approaches.


\section{Software}

\subsection{TensorFlow}

TensorFlow is an open-source machine learning software library released at the end of 2015 \cite{tensorflow:article}. It is developed by the Google Brain Team. \\
It provides a Python interface for efficient C++ code. After some time, we found that Tensorflow is not well documented at the moment and does not gave the needed freedom to easily rewrite some core features of their Word2Vec implementation, for example manipulating the internal trained lookup table.


\subsection{DeepLearning4Java}

DeepLearning4Java (DL4J) is an open-source machine learning software library released by Skymind \cite{dl4j:article}. \\
It runs on their scientific computing engine ND4J which provides fast matrix operations. DL4J is completely written in Java and provides a lot of freedom to manipulate lookup tables and extend their Word2Vec methods to work on abstract object like vectors. The developers are active on Gitter and offer a lot of information on how to use certain parts of DL4J. 


\section{Experiment Setup}

\subsection{Generalization}

As described in chapter \ref{cha:background}, we use generalized Word2Vec approaches to find patterns in EHR data. We use the OSIM dataset and represent each EHR as a $7$ dimensional vector. This vector is comparable to a word in a normal Word2Vec approach and functions as the abstract object in our generalized Word2Vec approaches. \\

Because we are working with high dimensional data, most instances of OMOP are quite unique. This is mainly due to a combination of specific disease codes and time intervals. To find patterns which are more general applicable, we start with generalizing our data. With generalizing we mean that values for some attributes are projected into categories. For example, the time intervals are projected into $4$ categories. \\

It is easy to generalize concepts as time intervals and demographics, for example we can say that people between the age $0$ and $10$ belong to category A. This becomes complex for disease diagnoses as it is domain specific and requires a lot of knowledge to generalize those. We talk about our solution to this in section \ref{sec:mapping}. \\

To see the effect of our generalization, see table \ref{tab:general}. You can see the $3$ most common vectors from our dataset before and after the generalization. \\

\begin{table}[]
\centering

\label{tab:general}
\begin{tabular}{ll}
\cline{1-1}
\multicolumn{1}{|l|}{\textbf{Before Generalization}}                     & {\ul }                                   \\ \hline
\multicolumn{1}{|l|}{\textit{Vector}}                                    & \multicolumn{1}{l|}{\textit{Occurences}} \\ \hline
\multicolumn{1}{|l|}{{[}1956.0, 8532.0, 65.0, 5.00000701E8, 0.0, 3.0{]}} & \multicolumn{1}{l|}{17574}               \\ \hline
\multicolumn{1}{|l|}{{[}1954.0, 8532.0, 65.0, 5.00000701E8, 0.0, 3.0{]}} & \multicolumn{1}{l|}{17536}               \\ \hline
\multicolumn{1}{|l|}{{[}1955.0, 8532.0, 65.0, 5.00000701E8, 0.0, 3.0{]}} & \multicolumn{1}{l|}{17476}               \\ \hline
                                                                         &                                          \\ \cline{1-1}
\multicolumn{1}{|l|}{\textbf{After Generalization}}                      &                                          \\ \hline
\multicolumn{1}{|l|}{\textit{Vector}}                                    & \multicolumn{1}{l|}{\textit{Occurences}} \\ \hline
\multicolumn{1}{|l|}{{[}6.0, 8532.0, 65.0, 784955.0, 1.0, 3.0{]}}        & \multicolumn{1}{l|}{282086}              \\ \hline
\multicolumn{1}{|l|}{{[}7.0, 8532.0, 65.0, 784955.0, 1.0, 3.0{]}}        & \multicolumn{1}{l|}{235459}              \\ \hline
\multicolumn{1}{|l|}{{[}5.0, 8532.0, 65.0, 784955.0, 1.0, 3.0{]}}        & \multicolumn{1}{l|}{230216}              \\ \hline
\end{tabular}

\caption{Three most common vectors in our dataset before and after generalization}
\end{table}


\subsection{Disease Code Mapping}
\label{sec:mapping}

In the section we discuss the method to generalize the disease codes used to label the diagnoses in the OMOP dataset. \\



-mapping -->TEST


\section{Results}
-what to test

\section{Conclusion}


http://omop.org/OSIM2


%%% Local Variables: 
%%% mode: latex
%%% TeX-master: "thesis"
%%% End: 
