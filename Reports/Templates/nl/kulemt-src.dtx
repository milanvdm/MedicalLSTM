% ^^A General remark:
% ^^A  Don't forget to adjust the version number of the class!
% \iffalse meta-comment
%
% Copyright (C) 2012 by Luc Van Eycken <Luc.VanEycken@esat.kuleuven.be>
% ---------------------------------------------------------------------
% 
% This file may be distributed and/or modified under the
% conditions of the LaTeX Project Public License, either version 1.2
% of this license or (at your option) any later version.
% The latest version of this license is in:
%
%    http://www.latex-project.org/lppl.txt
%
% and version 1.2 or later is part of all distributions of LaTeX 
% version 1999/12/01 or later.
%
% \fi
%
% \iffalse
%<*driver>
\ProvidesFile{kulemt-src.dtx}
%</driver>
%<class>\NeedsTeXFormat{LaTeX2e}[2001/06/01]
%<class>\ProvidesClass{kulemt}
%<*driver|class>
    [2015/05/25 v1.8
%</driver|class>
%<class>     KU Leuven engineering master thesis document class]
%<*driver>
     kulemt source]
%</driver>
%
%<*driver>
\documentclass{ltxdoc}
%% Steal \autoindex  and use it as \index, so hypdoc can redefine it
\newcommand*\autoindex{\index}
\newcommand*\UsageIndex[1]{\autoindex{#1\encapchar usage}}
\IfFileExists{hypdoc.sty}%
 {\usepackage{hypdoc}[2006/06/01]%
  \hypersetup{linkcolor=[rgb]{0,0,1}}}%
 {\newcommand\pdfstringdefDisableCommands[1]{}}
\IfFileExists{microtype.sty}{\usepackage{microtype}}{}
\newcommand*\NeedLineSpace[1]{%
  \vskip 0pt plus #1\baselineskip
  \penalty -100\vskip 0pt plus -#1\baselineskip
  \vskip #1\baselineskip
  \penalty 9999\vskip-#1\baselineskip}
\setlength\MacroIndent{1.5em}
\EnableCrossrefs         
\CodelineIndex
\RecordChanges
\setcounter{IndexColumns}{2}
\begin{document}
  \DocInput{kulemt-src.dtx}
\end{document}
%</driver>
% \fi
%
% \CheckSum{1681}
%
% \CharacterTable
%  {Upper-case    \A\B\C\D\E\F\G\H\I\J\K\L\M\N\O\P\Q\R\S\T\U\V\W\X\Y\Z
%   Lower-case    \a\b\c\d\e\f\g\h\i\j\k\l\m\n\o\p\q\r\s\t\u\v\w\x\y\z
%   Digits        \0\1\2\3\4\5\6\7\8\9
%   Exclamation   \!     Double quote  \"     Hash (number) \#
%   Dollar        \$     Percent       \%     Ampersand     \&
%   Acute accent  \'     Left paren    \(     Right paren   \)
%   Asterisk      \*     Plus          \+     Comma         \,
%   Minus         \-     Point         \.     Solidus       \/
%   Colon         \:     Semicolon     \;     Less than     \<
%   Equals        \=     Greater than  \>     Question mark \?
%   Commercial at \@     Left bracket  \[     Backslash     \\
%   Right bracket \]     Circumflex    \^     Underscore    \_
%   Grave accent  \`     Left brace    \{     Vertical bar  \|
%   Right brace   \}     Tilde         \~}
%
%
% ^^A \changes{v0.9}{2010/01/06}{Beta version 1}
% \changes{v1.0}{2010/03/01}{Initial release}
%
% \GetFileInfo{kulemt-src.dtx}
%
% ^^A Do not index standard TeX commands
% \DoNotIndex{\active,\advance,\baselineskip,\begingroup,\bigskip,\catcode,
%   \centerline,\csname,\def,\do,\edef,\else,\endcsname,\endgroup,\expandafter,
%   \fi,\gdef,\global,\hbox,\hfill,\hsize,\hskip,\hss,\if,\ifcase,\ifnum,\ifx,
%   \ignorespaces,\let,\lineskip,\lowercase,\month,\newcount,\newdimen,
%   \newif,\noexpand,\noindent,\null,\or,\par,\parindent,\parskip,\relax,
%   \space,\string,\the,\vbox,\vfill,\vskip,\vss,\xdef,\year}
% ^^A Do not index well known standard LaTeX commands
% \DoNotIndex{\,,\",\\,\arabic,\bfdefault,\centering,\chapter,
%   \ClassError,\ClassWarning,\ClassWarningNoLine,\clearpage,\copyright,
%   \CurrentOption,\end,\fboxsep,\fontfamily,\fontseries,\fontsize,
%   \i,\IfFileExists,\itshape,\Large,\medskip,\medskipamount,\MakeLowercase,
%   \MessageBreak,\newcommand,\newenvironment,\pagestyle,\PackageWarningNoLine,
%   \protect,\providecommand,\ProvidesFile,\raggedleft,\raggedright,
%   \renewcommand,\renewenvironment,\scshape,\selectfont,\setlength,\sloppypar,
%   \smallskipamount,\stretch,\textit,\textmd,\texttt,\textwidth,\ttdefault,
%   \upshape}
% ^^A "\ " needs special attention
% \let\origMakePrivateLetters\MakePrivateLetters
% \def\MakePrivateLetters{\origMakePrivateLetters \catcode`\ 12\relax}
% \DoNotIndex{\ }
% \let\MakePrivateLetters\origMakePrivateLetters
% ^^A Do not index internal (La)TeX commands
% \DoNotIndex{\@car,\@classoptionslist,\@currext,\@currname,\@empty,
%   \@firstofone,\@firstoftwo,\@for,\@gobble,\@hangfrom,\@ifnextchar,
%   \@ifundefined,\@ifpackageloaded,\@makeother,\@namedef,\@ne,\@nil,
%   \@onlypreamble,\@plus,\@secondoftwo,\@tempa,\@tempb,\@tempc,\@tempd,
%   \@tempcnta,\@tempcntb,\@tempdima,\@tfor,\@undefined,\@unusedoptionlist,
%   \@xiipt,\@xivpt,\@xviipt,\@xxvpt,\cf@encoding,\f@encoding,\g@addto@macro,
%   \ifin@,\in@,\m@ne,\p@,\protected@edef,\toks@,\z@,\z@skip,\zap@space}
% ^^A Do not index common commands from some required classes/packages
% \DoNotIndex{\addto,\centerlastline,\colorbox,\iflanguage,\ifpdf,\textcolor}
% ^^A Do not index \end... commands
% \DoNotIndex{\endabstract}
% 
% \let\pkg\textsf
% \let\cls\textsf
% \let\pstyle\textsf
% \let\ncolor\textsf
% \pdfstringdefDisableCommands{^^A
%   \let\cls\@firstofone
%   \let\pkg\@firstofone
% }
% \newcommand*\subsubsubsection{\subsubsection*}
% \makeatletter
% \newcommand\DefineItem[3]{\par
%   \noindent \marginpar{\raggedleft\MacroFont
%     \@for\@tempa:=#3\do{\ifhmode\break\fi #2\@tempa}}^^A
%   \@for\@tempa:=#3\do{^^A
%     \UsageIndex{\@tempa\actualchar #2{\@tempa} (#1)}^^A
%     \UsageIndex{#1s:\levelchar\@tempa\actualchar #2{\@tempa}}}\ignorespaces}
% \newcommand\DefineItemNoIndex[2]{\par
%   \noindent \marginpar{\raggedleft\MacroFont
%     \@for\@tempa:=#2\do{\ifhmode\break\fi #1\@tempa}}\ignorespaces}
% \makeatother
% \newcommand\DefineColorName[1]{\DefineItem{color}\ncolor{#1}}
% \newcommand\DefinePageStyle[1]{\DefineItem{pagestyle}\pstyle{#1}}
% \newcommand\DefineOptions[3]{^^A
%   \DefineItem{option}\texttt{#1}^^A
%   \DefineOptionsStart{#2}{#3}}
% \newcommand\DefineOptionsNoIndex[3]{^^A
%   \DefineItemNoIndex\texttt{#1}^^A
%   \DefineOptionsStart{#2}{#3}}
% \newcommand\DefineOptionsStart[2]{^^A
%   \begingroup
%     \let\oldmeta\meta \def\meta##1{\textrm{\oldmeta{##1}}}^^A
%     \def\or{\unskip\textrm{'' or ``}\ignorespaces}^^A
%     Option ``\texttt{#1}''.^^A
%   \endgroup
%   \if !#2!\else \qquad(#2)\fi
%   \hfil\penalty-999 \ignorespaces}
%
% \title{The source of the \cls{kulemt} class\thanks{This document
%     corresponds to \textsf{kulemt}~\fileversion, dated \filedate.}}
% \author{Luc Van Eycken \\ \texttt{Luc.VanEycken@esat.kuleuven.be}}
% \maketitle
%
% \begin{abstract}
% \noindent The \cls{kulemt} class provides a general \LaTeX\ class to
% typeset a KU~Leuven master thesis. The defaults are based on the
% requirements of the Faculty of Engineering Science, but the class can be
% configured and extended to suit other requirements.
% \end{abstract}
%
% \tableofcontents
%
% \section{Using the \cls{kulemt} class}
% This document describes the source of the \cls{kulemt} class and its
% default configuration file. The user manual of the class is available in
% a separate document |kulemt.pdf|, which serves as an example of the class
% at the same time. Using these names for the files, guarantees that
% \begin{verbatim}
%    > texdoc kulemt
%\end{verbatim}
% will find the user manual first.
%
% The \cls{kulemt} class is derived from the \cls{memoir} class. This class
% has the advantage that it includes the functionality of the most useful
% \LaTeX\ packages. Therefore it requires that the \cls{memoir} class as
% well as some required packages (\pkg{babel}, \pkg{helvet},
% \pkg{hyperref}, \pkg{keyval}, \pkg{mathpazo}, \pkg{mathptmx},
% \pkg{graphicx}, and \pkg{color}) are installed on the system.
% Besides the required packages, some additional image files are used,
% which are distributed with this class: the KU~Leuven logo (|logokul|) and
% the combined logo of the Faculty of Engineering Science and  the KU~Leuven
% (|logokuleng|). All image files are available as |.eps| (for PostScript
% printing) and as |.pdf| (for PDF generation).
%
% \DescribeMacro{\setup}
% Options for the \cls{kulemt} class are given using the
% ``\meta{key}|=|\meta{value}'' format from the \pkg{keyval} package. But
% some key-value pairs can't be used as options to \cls{kulemt} because
% options can't contain spaces, braces, or expandable commands. So we
% provide the \cs{setup}\marg{options} command to handle values without
% these restrictions. Of course it can be used for other keys too. In
% addition we allow multiple usage of \cs{setup}, but only in the document
% preamble.
%
% \DescribeEnv{preface}
% The |preface| environment typesets the preface page. The environment has
% one optional argument: the preface author. It defaults to the value of
% the |author| option. The argument can be used to remove the preface
% author or add a date to it.
%
% ^^A \changes{v0.91}{2010/01/06}{`\texttt{abstract*}' added}
% \NeedLineSpace2
% \DescribeEnv{abstract}
% \DescribeEnv{abstract*}
% The |abstract| environment typesets an abstract page in the default
% language. The |abstract*| environment does the same, but it uses the
% optional argument, defaulting to the official master language.
%
% ^^A \changes{v0.94}{2010/01/16}{`\cs{listoffiguresandtables}' and
% ^^A   `\cs{listfiguresandtablesname}' added}
% \NeedLineSpace2
% \DescribeMacro{\listoffiguresandtables}
% \DescribeMacro{\listfiguresandtablesname}
% Normally all ``List of \ldots'' overviews are printed on a separate page.
% However, for shorter texts like a master thesis these list may be smaller
% than half a page. Therefore an additional command
% \cs{listoffiguresandtables} is provided, which combines the list of
% figures and tables without a page break. The command
% \cs{listfiguresandtablesname} holds the title of that document section.
%
% \section{Configuring and extending the class}
% The \cls{kulemt} class is designed to automatically handle the
% typesetting of any KU~Leuven engineering master thesis. But some
% masters have additional requirements besides the common requirements from
% the faculty. Furthermore this class could also be used to typeset a
% master thesis of other faculties or of inter-faculty masters.
%
% The configuration file |kulemt.cfg| holds all information which is master
% or faculty dependent. The class itself provides most of the defaults for
% the Faculty of Engineering Science, except for the definition of the
% master data. The default configuration file for the Faculty of
% Engineering Science is found in section~\ref{sec:cfgfile}. Additionally
% the configuration file can be used to provide different defaults for the
% options. It can also be used to redefine commands to print data, such as
% the commands which provides faculty defaults (see page~\pageref{sec:comdef}).
%
% If a master wants more control over the typesetting, it can define its
% own configuration file, which is loaded after the general configuration
% file. The default name of this configuration file is
% `\texttt{kulemt-}\meta{id}\texttt{.cfg}'. The \meta{id} is the
% abbreviated master name as defined in the main configuration file, e.g.,
% `\texttt{arc}' for the `Master in de ingenieurswetenschappen:
% architectuur'. A typical use of this master specific configuration file
% is redefining the \cs{kulemt@fac@logo} command to add master specific logos.
%
% A master can also define its own class based on the \cls{kulemt} class.
% The new class can decide if a configuration file is used or not, and its
% name (see page~\pageref{def:cfgfile}). It can undefine existing options
% and define new ones. The command \cs{setup} can be used in the document
% preamble to set or modify the existing options as well as the new options.
%
%
% \StopEventually{\PrintChanges\PrintIndex}
%
% \section{Implementation of the \cls{kulemt} class}
% \iffalse
%<*class>
% \fi
% The namespace |kulemt| is claimed, so all commands are prefixed with
% |kulemt@| to avoid name clashes. In case you notice that other packages
% use this prefix too, please contact the author of this class!
%
% \subsection{Shorthand commands}
% We start by defining some shorthand commands to save \TeX\ memory tokens.
%
% \begin{macro}{\kulemt@cls}
% \cs{kulemt@cls} $\rightarrow$ name of this class
%    \begin{macrocode}
\def\kulemt@cls{kulemt}
%    \end{macrocode}
% \end{macro}
% 
% \begin{macro}{\kulemt@error}
% \cs{kulemt@error}\marg{msg} signals a fatal error with message \meta{msg}.
%    \begin{macrocode}
\def\kulemt@error#1{%
  \ClassError\kulemt@cls{#1}{Exit, correct this error and rerun.}}
%    \end{macrocode}
% \end{macro}
%
% \begin{macro}{\kulemt@opt@missingpkg}
% \cs{kulemt@opt@missingpkg}\marg{opt}\marg{pkg} signals a fatal error
% indicating that option \meta{opt} can only be used if package \meta{pkg}
% is installed.
%    \begin{macrocode}
\def\kulemt@opt@missingpkg#1#2{\kulemt@error{%
    The option `#1' is ignored because\MessageBreak
    it requires the installation of the package `#2'}}
%    \end{macrocode}
% \end{macro}
% 
% \begin{macro}{\kulemt@ifdutch}
% \cs{kulemt@ifdutch}\marg{D}\marg{E} $\equiv$
%   \cs{iflanguage}|{dutch}|\marg{D}\marg{E}\\
% Note: This command is robust.
%    \begin{macrocode}
\def\kulemt@ifdutch{\protect\iflanguage{dutch}}
%    \end{macrocode}
% \end{macro}
% 
% \changes{v1.4}{2011/06/07}{Raise an error if language \texttt{dutch} is
%   not installed}^^A
% This class only works if the language |dutch| is available. We raise an
% error now if it is missing. We don't test for |english|, since this
% language is the default in a TeX installation.
%    \begin{macrocode}
\@ifundefined{l@dutch}{\kulemt@error{%
    The language `dutch' is not available.\MessageBreak
    You must install Dutch hyphenation patterns}}{}
%    \end{macrocode}
%
% \begin{macro}{\kulemt@ifand}
% \changes{v1.4}{2011/06/07}{Take 3 arguments and make it expandable}
% \begin{macro}{\kulemt@ifand@}
% The command \cs{kulemt@ifand}\marg{cs}\marg{true}\marg{false} checks for
% the presence of \cs{and} in \meta{cs}. If \cs{and} is found, \meta{true}
% is executed, else \meta{false} is executed.
% To make this command fully expandable, we assume that \cs{and} is never
% followed by a |=|.
%    \begin{macrocode}
\def\kulemt@ifand#1{\expandafter\kulemt@ifand@ #1\and\@nil}
\def\kulemt@ifand@#1\and #2\@nil{%
  \if=\noexpand#2=\expandafter\@secondoftwo\else
    \expandafter\@firstoftwo\fi}
%    \end{macrocode}
% \end{macro}
% \end{macro}
%
% \subsection{Options}
% \subsubsection{Option handling commands}
% Option handling is based on the standard \pkg{keyval} package with ideas
% coming from the \pkg{kvoptions} package.
%    \begin{macrocode}
\RequirePackage{keyval}
%    \end{macrocode}
%
% \begin{macro}{\setup}
% Keys can be used as options or in the argument \meta{arg} of
% \cs{setup}\marg{arg}. Some keys only make sense as an option because they
% are used when loading the \cls{kulemt} class. On the other hand some
% key-value pairs can't be used as options because the value contains
% expandable commands. Finally, some keys can only be defined once without
% problems. The command \cs{kulemt@do@once@opts} guarantees this.
%
% ^^A \changes{v0.99}{2010/02/27}{Set the language extra definitions when
% ^^A   parsing the argument and make the double quote active for Dutch}
% The \cs{setup} command first calls \cs{kulemt@catcode@setup} to set the
% correct catcodes before parsing its argument.
%    \begin{macrocode}
\newcommand*\setup{%
  \kulemt@catcode@setup
  \kulemt@setup}
\@onlypreamble\setup
%    \end{macrocode}
% \begin{macro}{\kulemt@setup}
% The command \cs{kulemt@setup} does the actual handling of the argument of
% \cs{setup}. After using the argument it restores the original catcodes
% via \cs{kulemt@uncatcode@setup}.
%    \begin{macrocode}
\def\kulemt@setup#1{%
  \setkeys{kulemt}{#1}%
  \kulemt@uncatcode@setup
  \kulemt@do@once@opts}
%    \end{macrocode}
% \end{macro}
% \end{macro}
%
% \begin{macro}{\kulemt@catcode@setup}
% The command \cs{kulemt@catcode@setup} can be used to change the catcodes
% before handling the \cs{setup} parameters. By default it defines the
% language shorthands, such as `|"|' in Dutch. But unfortunately the
% \pkg{babel} package doesn't make the shorthand characters active in the
% preamble, so we have to do it ourselves.
%    \begin{macrocode}
\def\kulemt@catcode@setup{%
  \csname extras\languagename\endcsname
  \kulemt@ifdutch{\catcode`\"\active}{}}
%    \end{macrocode}
% \end{macro}
% \begin{macro}{\kulemt@uncatcode@setup}
% The command \cs{kulemt@uncatcode@setup} reverses the catcode changes
% introduced by \cs{kulemt@catcode@setup}. By default it undefines the
% language shorthands and turns the active characters into normal ones.
%    \begin{macrocode}
\def\kulemt@uncatcode@setup{%
  \csname noextras\languagename\endcsname
  \@makeother\"}
%    \end{macrocode}
% \end{macro}
%
% \begin{macro}{\kulemt@invalidate@key}
% The \cs{kulemt@invalidate@key}\marg{key}\marg{how-used} command
% invalidates the key \meta{key} by (re)defining it as a class warning.
% The parameter \meta{how-used} tells how the key could be used. Typical
% examples of \meta{how-used} are `|once|' or `|as |\ldots'.
%    \begin{macrocode}
\def\kulemt@invalidate@key#1#2{%
  \define@key{kulemt}{#1}{\ClassWarning\kulemt@cls{%
      The key `#1' can only be used #2.\MessageBreak
      It is ignored}}}
%    \end{macrocode}
% \end{macro}
%
% \begin{macro}{\kulemt@keynovalue}
% The \cs{kulemt@keynovalue}\marg{key}\marg{definition} command defines the
% \meta{key} option without value. If a value is given anyway, it is simply
% ignored.
%    \begin{macrocode}
\def\kulemt@keynovalue#1#2{%
  \define@key{kulemt}{#1}[]{%
    \def\@tempa{##1}\ifx\@tempa\@empty\else
      \PackageWarningNoLine\kulemt@cls{Value of option `#1' ignored}\fi
    #2}}
%    \end{macrocode}
% \end{macro}
%
% \begin{macro}{\kulemt@clskey}
% \begin{macro}{\kulemt@clskeys}
% The command \cs{kulemt@clskey} is used to define a key, which can only be used
% as a class option.
% A list of these keys is kept in the command sequence \cs{kulemt@clskeys}.
% This can be used to invalidate them at the appropriate moment.
%    \begin{macrocode}
\def\kulemt@clskeys{}
%    \end{macrocode}
% \end{macro}
% The \cs{kulemt@clskey}\marg{key}\oarg{def-value}\marg{definition} command
% defines the key \meta{key} to call the \meta{definition}. If the optional
% parameter is used, the key can be used without value and gets the
% \meta{def-value} value then. See the \pkg{keyval} package for more
% information.
%    \begin{macrocode}
\def\kulemt@clskey#1{%
  \edef\kulemt@clskeys{\kulemt@clskeys,#1}%
  \define@key{kulemt}{#1}}
%    \end{macrocode}
% \end{macro}
%
% \begin{macro}{\kulemt@clsopt}
% The \cs{kulemt@clsopt}\marg{key}\marg{definition} command defines the
% class option \meta{key} without value. If a value is given anyway, it is
% simply ignored.
%    \begin{macrocode}
\def\kulemt@clsopt#1{%
  \edef\kulemt@clskeys{\kulemt@clskeys,#1}%
  \kulemt@keynovalue{#1}}
%    \end{macrocode}
% \end{macro}
%
% \begin{macro}{\kulemt@process@ptions}
% The command \cs{kulemt@process@ptions} is used to handle all options
% which were defined as options. Its definition is heavily inspired by the
% command \cs{ProcessKeyvalOptions} of the \pkg{kvoptions} package. A
% simplified version is included here to remove a dependency on a package
% which is not guaranteed to be present.
%    \begin{macrocode}
\def\kulemt@process@ptions{%
  \@ifundefined{opt@\@currname.\@currext}{}%
   {\begingroup
    \toks@\expandafter\expandafter\expandafter{%
      \csname opt@\@currname.\@currext\endcsname}%
    \edef\CurrentOption{\the\toks@}%
    \toks@{}%
    \@for\CurrentOption:=\CurrentOption\do{%
      \@ifundefined{%
        KV@kulemt@\expandafter\kulemt@getkey\CurrentOption=\@nil}%
%    \end{macrocode}
% Options with unknown keys are put in the unused option list.
%    \begin{macrocode}
       {\ifx\@unusedoptionlist\@empty
          \global\let\@unusedoptionlist\CurrentOption
        \else
          \expandafter\expandafter\expandafter\gdef
          \expandafter\expandafter\expandafter\@unusedoptionlist
          \expandafter\expandafter\expandafter{%
            \expandafter\@unusedoptionlist
            \expandafter,\CurrentOption}%
        \fi}%
%    \end{macrocode}
% Options with known keys are temporarily stored in \cs{toks@}. The
% problems with braces around a value are reduced with
% \cs{kulemt@update@classoptions}.
%    \begin{macrocode}
       {\toks@\expandafter{%
          \the\expandafter\toks@\expandafter,\CurrentOption}%
        \expandafter\kulemt@update@classoptions\CurrentOption=aa=\@nil
       }}%
    \edef\@tempa{\endgroup
      \noexpand\setkeys{kulemt}{\the\toks@}}%
    \@tempa
    \let\CurrentOption\@empty}%
%    \end{macrocode}
% After processing the class options, invalidate all class options so they
% can't be used in subsequent \cs{setup} commands.
%    \begin{macrocode}
  \@for\@tempa:=\kulemt@clskeys\do{%
    \expandafter\kulemt@invalidate@key\expandafter{\@tempa}{%
      as a class option}}}
%    \end{macrocode}
% \begin{macro}{\kulemt@getkey}
% The \cs{kulemt@getkey} command is a copy of \cs{KVO@getkey} of the
% \pkg{kvoptions} package. It used inside \cs{kulemt@process@ptions} to get
% the key from a key-value pair.
%    \begin{macrocode}
\def\kulemt@getkey#1=#2\@nil{#1}
%    \end{macrocode}
% \end{macro}
% \end{macro}
%
% \begin{macro}{\kulemt@update@classoptions}
% The \cs{kulemt@update@classoptions} command removes the first key-value
% pair with a value of exactly one token from \cs{@classoptionslist}. We
% assume we don't have single letter values so we can use it remove pairs
% with a value surrounded by braces. We also assume that these key-value
% options are only used in the class itself and don't have to be kept as
% global options for other packages. As you see, we make a lot of
% assumptions here!
%    \begin{macrocode}
\def\kulemt@update@classoptions#1=#2#3=#4\@nil{%
  \def\@tempa{#3}\ifx\@tempa\@empty
    \def\@tempa##1,#1=##2,##3\@nil{##1,##3\@nil}%
    \def\@tempb,##1,\@nil{##1}%
    \xdef\@classoptionslist{\expandafter\expandafter\expandafter\@tempb
      \expandafter\@tempa\expandafter,\@classoptionslist,\@nil}%
  \fi}
%    \end{macrocode}
% \end{macro}
%
% \subsubsection{Keys which can only be used as class options}
% The following keys can only be used as class options because they are
% either used directly in this class file or they must be passed as class
% options to \cls{memoir}. These options are defined with \cs{kulemt@clskey}.
%
% \subsubsubsection{Selecting the master}
% \DefineOptions{master}{master=\meta{id}}{required option}
% The \meta{id} defines the master. The set of allowed \meta{id}s is
% defined later in the configuration file.
%    \begin{macrocode}
\kulemt@clskey{master}{\lowercase{\edef\kulemt@opt@master{#1}}}
%    \end{macrocode}
% \begin{macro}{\kulemt@opt@master}
% The \meta{id} is lowercased and saved in \cs{kulemt@opt@master}.
%    \begin{macrocode}
\def\kulemt@opt@master{}
%    \end{macrocode}
% \end{macro}
%
% \subsubsubsection{Type size}
% \DefineOptions{10pt,11pt}{10pt \or 11pt}{default is \texttt{11pt}}
% These two options are mutually exclusive.
%    \begin{macrocode}
\kulemt@clsopt{10pt}{\def\kulemt@ptsize{0}}
\kulemt@clsopt{11pt}{\def\kulemt@ptsize{1}}
%    \end{macrocode}
% \begin{macro}{\kulemt@ptsize}
% The last digit of the type size is stored in the command \cs{kulemt@ptsize}.
% It is initialized with the default value (|11pt|). It will be passed to
% \cls{memoir} but will also be used to determine the page layout.
%    \begin{macrocode}
\def\kulemt@ptsize{1}
%    \end{macrocode}
% \end{macro}
%
% \subsubsubsection{Printing options}
% \DefineOptions{openright,openleft,openany}{^^A
%   openright \or openleft \or openany}{default is \texttt{openright}}
% These three options are mutually exclusive.
% They determine on which page a chapter starts: recto, verso or any page.
% These options are passed directly to the \cls{memoir} class.
%    \begin{macrocode}
\kulemt@clsopt{openright}{\PassOptionsToClass{open}{memoir}}
\kulemt@clsopt{openany}{\PassOptionsToClass{openany}{memoir}}
\kulemt@clsopt{openleft}{\PassOptionsToClass{openleft}{memoir}}
%    \end{macrocode}
%
% \DefineOptions{oneside,twoside}{^^A
%   oneside \or twoside}{default is \texttt{twoside}}
% These two options are mutually exclusive.
% They indicate whether the document will be printed on one or both sides
% of the paper. When you prepare electronic documents, it makes sense to
% choose the |oneside| option.
%    \begin{macrocode}
\kulemt@clsopt{oneside}{\PassOptionsToClass{oneside}{memoir}}
\kulemt@clsopt{twoside}{\PassOptionsToClass{twoside}{memoir}}
%    \end{macrocode}
%
% \DefineOptions{bind}{bind=\meta{dimen}}{default \texttt{0pt}}
% This option specifies the loss \meta{dimen} of visible paper due to
% binding the book.
%    \begin{macrocode}
\kulemt@clskey{bind}{\setlength\kulemt@bind{#1}}
%    \end{macrocode}
% \begin{macro}{\kulemt@bind}
% The \meta{dimen} is saved in the register \cs{kulemt@bind}. It is
% initialized with the default value 0\,pt.
%    \begin{macrocode}
\newdimen\kulemt@bind
\kulemt@bind\z@
%    \end{macrocode}
% \end{macro}
%
% \DefineOptions{draft}{draft}{}
% The |draft| option is passed directly to the \cls{memoir} class. The effect
% is to mark overfull lines and to not show graphics content.
%    \begin{macrocode}
\kulemt@clsopt{draft}{\PassOptionsToClass{draft}{memoir}}
%    \end{macrocode}
%
% \subsubsubsection{Language options}
% These options pass information to the \pkg{babel} package, which is included
% by default.
% \begin{macro}{\kulemt@babel@opt}
% The options of the \pkg{babel} package are collected in the
% \cs{kulemt@babel@opt} command. At least the languages English and Dutch
% are initialized since they may be used on the title or copyright page.
%    \begin{macrocode}
\def\kulemt@babel@opt{english,dutch}
%    \end{macrocode}
% \end{macro}
% 
% \DefineOptions{dutch,english}{dutch \or english}{}
% These options allow you to select the main text language. Since you can
% have only one main text language, the two options are mutually exclusive.
%    \begin{macrocode}
\kulemt@clsopt{dutch}{\def\kulemt@language{dutch}}
\kulemt@clsopt{english}{\def\kulemt@language{english}}
%    \end{macrocode}
% \begin{macro}{\kulemt@language}
% The text language is stored in the command \cs{kulemt@language}.
% It is initialized with the master language, which will be stored in
% \cs{kulemt@master@language}.
%    \begin{macrocode}
\def\kulemt@language{\kulemt@master@language}
%    \end{macrocode}
% \end{macro}
% 
% \DefineOptions{extralanguage}{extralanguage=\meta{lang}}{}
% This option adds \meta{lang} to the \pkg{babel} options, but only if it
% wasn't included yet.
%    \begin{macrocode}
\kulemt@clskey{extralanguage}{%
  \edef\@tempa{{,#1,}{,\kulemt@babel@opt,}}\expandafter\in@\@tempa
  \ifin@\else \edef\kulemt@babel@opt{#1,\kulemt@babel@opt}\fi}
%    \end{macrocode}
%
% \subsubsubsection{Options to disable package loading}
% Some of the packages loaded below may be absent in a specific
% installation or they may conflict with other packages used in the
% document. Therefore we provide options to disable them in case of
% emergency.
%
% ^^A \changes{v0.93}{2010/01/15}{Option `\textsf{nomicrotype}' added}
% \DefineOptions{nomicrotype}{nomicrotype}{}
% This option disables the \pkg{microtype} package.
%    \begin{macrocode}
\kulemt@clsopt{nomicrotype}{\kulemt@microtypefalse}
%    \end{macrocode}
% \begin{macro}{\ifkulemt@microtype}
% The switch |kulemt@microtype| indicates whether the \pkg{microtype} must be
% loaded or not. By default it is loaded.
%    \begin{macrocode}
\newif\ifkulemt@microtype \kulemt@microtypetrue
%    \end{macrocode}
% \end{macro}
%
% \subsubsubsection{Other options}
% \DefineOptions{fleqn}{fleqn}{}
% The |fleqn| option is passed directly to the \cls{memoir} class. The effect
% is to flush equations left.
%    \begin{macrocode}
\kulemt@clsopt{fleqn}{\PassOptionsToClass{fleqn}{memoir}}
%    \end{macrocode}
%
% \subsubsection{Keys which can only be used once}
% The following keys can only be used once in the preamble, either as an
% option or once in \cs{setup}.
% \begin{macro}{\kulemt@do@once@opts}
% The \cs{kulemt@do@once@opts} command holds the commands to execute once,
% either after option processing later on or at the end of \cs{setup}.
%    \begin{macrocode}
\gdef\kulemt@do@once@opts{}
%    \end{macrocode}
% \end{macro}
% \begin{macro}{\kulemt@add@once@opts}
% The \cs{kulemt@add@once@opts}\marg{key}\marg{cs}\marg{coms} command appends
% \meta{coms} to \cs{kulemt@do@once@opts}, surrounded by a test. The
% commands \meta{coms} are only executed if the command \meta{cs} is not
% empty. When the commands \meta{coms} executed, the key \meta{key} is
% invalidated.
% \changes{v1.8}{2015/05/25}{Make sure \cs{@tempa} is not overwritten}
%    \begin{macrocode}
\def\kulemt@add@once@opts#1#2#3{\g@addto@macro\kulemt@do@once@opts{%
    \ifx #2\@empty\else
      \kulemt@invalidate@key{#1}{once}%
      \def\@tempa{#3\let#2\@empty}%
      \expandafter\@tempa
    \fi}}
%    \end{macrocode}
% \end{macro}
%
% \subsubsubsection{Setting the master option}
% \DefineOptions{masteroption}{masteroption=\meta{mo}}{default is no option}\\
% The \meta{mo} defines the master option or major topic (in Dutch
% ``optie'' or ``afstudeerrichting''). The \meta{mo} is a text starting
% with ``|option |\ldots'' (or ``|optie |\ldots'' or
% ``|afstudeerrichting |\ldots''). If the master defines options (see
% \cs{kulemt@def@master} for details) you can use the option abbreviation
% here.\\
% The \meta{mo} can be a comma separated list of options in case students
% of different options work on one common master thesis. If a comma is used
% inside a master option declaration, it must be hidden inside braces.
%    \begin{macrocode}
\define@key{kulemt}{masteroption}{\def\kulemt@opt@masteroption{#1}}
%    \end{macrocode}
% \begin{macro}{\kulemt@opt@masteroption}
% The \meta{mo} is saved in \cs{kulemt@opt@masteroption}.
%    \begin{macrocode}
\def\kulemt@opt@masteroption{}
%    \end{macrocode}
% \end{macro}
% \begin{macro}{\kulemt@master@option}
% At the end of \cs{setup} the command \cs{kulemt@master@option} is set. It
% holds a comma separated list of the expanded master option texts.
%    \begin{macrocode}
\def\kulemt@master@option{}
%    \end{macrocode}
% If the lowercased \cs{kulemt@opt@masteroption} is an abbreviation of an
% option specified in \cs{kulemt@master@options}, \cs{kulemt@master@option}
% is set to the full master option. Otherwise \cs{kulemt@master@option} is
% set to the original content of \cs{kulemt@opt@masteroption}.\\
% If \cs{kulemt@opt@masteroption} is a comma separated list, each item in
% the list is handled in the way described above.\\
% \changes{v1.7}{2013/05/01}{Handle disallowing master options.}^^A
% If a master does not want master options to be printed, raise an error if
% the option ``|masteroption|'' is used.
%    \begin{macrocode}
\kulemt@add@once@opts{masteroption}\kulemt@opt@masteroption{%
  \global\let\kulemt@master@option\@empty
  \ifx\kulemt@master@options@vl\relax
    \ifx\kulemt@opt@masteroption\@empty\else
      \kulemt@error{%
        The option `masteroption' is ignored because\MessageBreak
        your program disallows a master option on front pages}%
    \fi
  \else
    \@for\@tempa:=\kulemt@opt@masteroption\do{%
      \let\@tempc\@tempa
      \expandafter\kulemt@handle@mo\expandafter\kulemt@master@options
      \expandafter{%
        \expandafter\kulemt@add@mo@tempc\expandafter{\@tempa}}%
      \expandafter\g@addto@macro\expandafter\kulemt@master@option
        \expandafter{\@tempc,}}%
  \fi}
%    \end{macrocode}
% \end{macro}
% \begin{macro}{\kulemt@add@mo@tempc}
% The command \cs{kulemt@add@mo@tempc}\marg{smo}\marg{abbrev}\marg{full}
% adds to \cs{@tempc} \meta{full} if the lowercased \meta{smo} (a single
% master option) equals \meta{abbrev}.
%    \begin{macrocode}
\def\kulemt@add@mo@tempc#1#2#3{%
  \lowercase{\def\@tempa{#1}}\def\@tempb{#2}\ifx\@tempa\@tempb
    \def\@tempc{#3}\fi}
%    \end{macrocode}
% \end{macro}
% \begin{macro}{\kulemt@handle@mo}
% \changes{v1.6}{2012/05/13}{Extra first argument \meta{molist}}^^A
% The command \cs{kulemt@handle@mo}\marg{molist}\marg{func} processes each
% element from the token list \meta{molist}, such as
% \cs{kulemt@master@options}, with \meta{func}.
%    \begin{macrocode}
\def\kulemt@handle@mo#1#2{%
  \expandafter\@tfor\expandafter\@tempd\expandafter:\expandafter=#1\do{%
    \expandafter\kulemt@handle@mo@\@tempd\@nil{#2}}}
%    \end{macrocode}
% \end{macro}
% \begin{macro}{\kulemt@handle@mo@}
% Helper command which calls \meta{func}\marg{abbrev}\marg{full}.
%    \begin{macrocode}
\def\kulemt@handle@mo@ #1:#2\@nil #3{#3{#1}{#2}}
%    \end{macrocode}
% \end{macro}
%
% \subsubsubsection{Input file encoding}
% \DefineOptions{inputenc}{inputenc=\meta{enc}}{}
% This option specifies the character encoding \meta{enc} of the document.
% The \meta{enc} must be a valid option of the \pkg{inputenc} package, e.g.,
% |latin1| or |utf8|.
% \changes{v1.7}{2013/04/12}{Explicitly set input encoding to \texttt{ascii}}^^A
% If this option is not used, the document is supposed to use the ASCII
% character encoding. To detect another encoding, we set the input encoding
% initially to |ascii|.
%    \begin{macrocode}
\RequirePackage[ascii]{inputenc}
\define@key{kulemt}{inputenc}{\def\kulemt@opt@inputenc{#1}}
%    \end{macrocode}
% \begin{macro}{\kulemt@opt@inputenc}
% The \meta{enc} is saved in \cs{kulemt@opt@inputenc}.
%    \begin{macrocode}
\def\kulemt@opt@inputenc{}
%    \end{macrocode}
% At the end of \cs{setup} the input encoding is changed.
%    \begin{macrocode}
\kulemt@add@once@opts{inputenc}\kulemt@opt@inputenc{%
  \inputencoding\kulemt@opt@inputenc}
%    \end{macrocode}
% \end{macro}
%
% \subsubsubsection{Fonts}
% \DefineOptions{font}{^^A
%   font=\meta{fnt} \or font=\meta{fnt}:\meta{fntopt}}{default is \texttt{cm}}\\
% The key |font| lets you specify the font family \meta{fnt} to use. A font
% family consists of a serif font, a sans-serif font, a typewriter font
% and a math font. In our implementation we also set the font encoding.
% The comma separated list \meta{fntopt} holds the options to pass to the
% font definition package. This implies that the possibilities and their
% meaning depends on the \meta{fnt}.
%    \begin{macrocode}
\define@key{kulemt}{font}{%
  \edef\@tempa{\zap@space#1 \@empty}%
  \expandafter\kulemt@set@font\@tempa::\@empty}
%    \end{macrocode}
% \begin{macro}{\kulemt@set@font}
% The \cs{kulemt@set@font} command parses ``\meta{fnt}:\meta{fntopt}'' and
% calls the handler \cs{kulemt@set@font@}\meta{fnt}\marg{fntopt}. If
% the handler is not defined, an error is signaled.
%    \begin{macrocode}
\def\kulemt@set@font#1:#2:#3\@empty{%
  \expandafter\let\expandafter\@tempa
    \csname kulemt@set@font@#1\endcsname
  \ifx\@tempa\relax
    \kulemt@error{Font `#1' is ignored because it is an unknown font}%
  \else \@tempa{#2}\fi}
%    \end{macrocode}
% \end{macro}
% \begin{macro}{\kulemt@loadfont@}
% Every handler must define \cs{kulemt@loadfont@} to hold the commands to
% define the different fonts.
%    \begin{macrocode}
\let\kulemt@loadfont@\@empty
%    \end{macrocode}
% \end{macro}
% At the end of \cs{setup} the fonts are loaded, followed by the
% \pkg{microtype} package.\label{loadmt-after-font}
%    \begin{macrocode}
\kulemt@add@once@opts{font}\kulemt@loadfont@{%
  \kulemt@loadfont@ \kulemt@loadmt}
%    \end{macrocode}
%
% \noindent The font options described below show the implemented font
% families. \par \smallskip
%
% \DefineOptionsNoIndex{font=cm}{font=cm}{}\\
% Use the Computer Modern family: |cmr| (serif), |cmss| (sans-serif),
% |cmtt| (typewriter), and Computer Modern math. OT1 is used as character
% encoding.
% \begin{macro}{\kulemt@set@font@cm}
% Since these are the default \LaTeX\ fonts, nothing special has to be done.
%    \begin{macrocode}
\def\kulemt@set@font@cm#1{\let\kulemt@loadfont@\relax}
%    \end{macrocode}
% \end{macro}
%
% \DefineOptionsNoIndex{font=lm}{font=lm}{}\\
% Use the Latin Modern family: |lmr| (serif), |lmss| (sans-serif),
% |lmtt| (typewriter), and Latin Modern math. T1 is used as character
% encoding.
% \begin{macro}{\kulemt@set@font@lm}
% On older systems, Latin Modern is not installed by default, so we have to
% check for it.
%    \begin{macrocode}
\def\kulemt@set@font@lm#1{%
  \IfFileExists{lmodern.sty}%
   {\def\kulemt@loadfont@{%
      \RequirePackage[T1]{fontenc}%
      \RequirePackage{lmodern}}}%
   {\kulemt@opt@missingpkg{font=lm}{lm}}}
%    \end{macrocode}
% \end{macro}
%
% \DefineOptionsNoIndex{font=palatino}{^^A
%   font=palatino \or font=palatino:\meta{mpopts}}{}\\
% Use Palatino as serif font, Helvetica as sans-serif, |lmtt| (if
% available) as typewriter, and Pazo math. T1 is used as character
% encoding. For possible \meta{mpopts} options to the \pkg{mathpazo} package,
% see its documentation. \\
% Note: if the FPL fonts are installed use |sc| as part of \meta{mpopts} to
% get real small caps.
% \begin{macro}{\kulemt@set@font@palatino}
% Helvetica is scaled down to fit the x-height of Palatino.
%    \begin{macrocode}
\def\kulemt@set@font@palatino#1{%
  \def\kulemt@loadfont@{%
    \RequirePackage[T1]{fontenc}%
    \RequirePackage[#1]{mathpazo}%
    \RequirePackage[scaled=.9]{helvet}}%
  \IfFileExists{lmodern.sty}{%
    \g@addto@macro\kulemt@loadfont@{%
      \renewcommand{\ttdefault}{lmtt}}}{}}
%    \end{macrocode}
% \end{macro}
%
% \DefineOptionsNoIndex{font=times}{font=times \or font=times:\meta{mtopts}}{}\\
% Use Times as serif font, Helvetica as sans-serif, |lmtt| (if
% available) as typewriter, and the virtual `mathptmx' fonts as math font.
% T1 is used as character encoding. For possible \meta{mtopts} options to
% the \pkg{mathptmx} package, see its documentation.\\
% Note: This implementation has no boldmath version!
% \begin{macro}{\kulemt@set@font@times}
% Helvetica is scaled down to fit the x-height of Times.
%    \begin{macrocode}
\def\kulemt@set@font@times#1{%
  \def\kulemt@loadfont@{%
    \RequirePackage[T1]{fontenc}%
    \RequirePackage[#1]{mathptmx}%
    \RequirePackage[scaled=.9]{helvet}}%
  \IfFileExists{lmodern.sty}{%
    \g@addto@macro\kulemt@loadfont@{%
      \renewcommand{\ttdefault}{lmtt}}}{}}
%    \end{macrocode}
% \end{macro}
%
% \DefineOptionsNoIndex{font=utopia}{^^A
%   font=utopia \or font=utopia:\meta{muopts}}{}\\
% Use Utopia as serif font, Helvetica as sans-serif, |lmtt| (if
% available) as typewriter, and the Fourier math font.
% T1 is used as character encoding. For possible \meta{muopts} options to
% the \pkg{fourier} package, see its documentation.
% \begin{macro}{\kulemt@set@font@utopia}
% This option requires the \pkg{fourier} package for the math fonts.
% Helvetica is scaled down to fit the x-height of Utopia.
%    \begin{macrocode}
\def\kulemt@set@font@utopia#1{%
  \IfFileExists{fourier.sty}%
    {\def\kulemt@loadfont@{%
       \RequirePackage[T1]{fontenc}%
       \RequirePackage[#1]{fourier}%
       \RequirePackage[scaled=.9]{helvet}}%
     \IfFileExists{lmodern.sty}{%
       \g@addto@macro\kulemt@loadfont@{%
         \renewcommand{\ttdefault}{lmtt}}}{}}%
    {\kulemt@opt@missingpkg{font=utopia}{fourier}}}
%    \end{macrocode}
% \end{macro}
%
% \subsubsection{Keys which can be used multiple times}
% The following keys can be used multiple times in the preamble, as an
% option and in every \cs{setup}.
%
% \subsubsubsection{Information for the title page}
% \DefineOptions{title}{title=\meta{title}}{required option}
% This option sets the title using the standard \LaTeX\ command \cs{title},
% which stores \meta{title} in \cs{@title}. Using \cs{title} and
% \cs{@title} also ensures that the \pkg{hyperref} package picks up the title.\\
% Since this is a required option, \cs{@title} is initialized with an error
% message.
%    \begin{macrocode}
\define@key{kulemt}{title}{\title{#1}}
\def\@title{\kulemt@error{No title given}}
%    \end{macrocode}
%
% \DefineOptions{subtitle}{subtitle=\meta{stitle}}{}
% This option specifies the subtitle \meta{stitle}.
%    \begin{macrocode}
\define@key{kulemt}{subtitle}{\def\kulemt@subtitle{#1}}
%    \end{macrocode}
% \begin{macro}{\kulemt@subtitle}
% The \meta{stitle} is saved in \cs{kulemt@subtitle}.
%    \begin{macrocode}
\def\kulemt@subtitle{}
%    \end{macrocode}
% \end{macro}
%
% \DefineOptions{author}{author=\meta{authors}}{required option}
% This option sets the authors using the standard \LaTeX\ command \cs{author},
% which stores \meta{authors} in \cs{@author}. Using \cs{author} and
% \cs{@author} also ensures that the \pkg{hyperref} package picks up the
% authors. If multiple authors are given, they should be separated by
% \cs{and}.\\
% Since this option is a required option, \cs{@author} is initialized with
% an error message.
%    \begin{macrocode}
\define@key{kulemt}{author}{\author{#1}}
\def\@author{\kulemt@error{No author given}}
%    \end{macrocode}
%
% \DefineOptions{promotor,promoter}{^^A
%   promotor=\meta{promoters}\or promoter=\meta{promoters}}{required}
% This option lists the promoter(s). If multiple promoters are given, they
% should be separated by \cs{and}.
% \changes{v1.2}{2010/08/03}{Disallow empty values for the promotor keyword}^^A
% No empty value is allowed since a promoter must be printed on the front pages.
%    \begin{macrocode}
\define@key{kulemt}{promotor}{%
  \def\@tempa{#1}\ifx\@tempa\@empty\else \def\kulemt@promotor{#1}\fi}
%    \end{macrocode}
% \changes{v1.7}{2013/05/01}{New option \texttt{promoter}.}^^A
% The option |promoter| is an alias to option |promotor|, so the correct
% English terminology can also be used.
%    \begin{macrocode}
\let\KV@kulemt@promoter\KV@kulemt@promotor
%    \end{macrocode}
% \begin{macro}{\kulemt@promotor}
% The \meta{promoters} is saved in \cs{kulemt@promotor}. Since the
% |promotor| option is a required option, the command is initialized with
% an error message.
%    \begin{macrocode}
\def\kulemt@promotor{\kulemt@error{No promoter given}}
%    \end{macrocode}
% \end{macro}
%
% \DefineOptions{assessor}{assessor=\meta{assessors}}{required option}
% This option lists the assessors, separated by \cs{and}.
%    \begin{macrocode}
\define@key{kulemt}{assessor}{\def\kulemt@assessor{#1}}
%    \end{macrocode}
% \begin{macro}{\kulemt@assessor}
% The \meta{assessors} is saved in \cs{kulemt@assessor}. Since the
% |assessor| option is a required option, the command is initialized with
% an error message.
%    \begin{macrocode}
\def\kulemt@assessor{\kulemt@error{No assessor given}}
%    \end{macrocode}
% \end{macro}
%
% \DefineOptions{assistant}{assistant=\meta{assistants}}{required option}
% This option lists the assistant(s). If multiple assistants are given, they
% should be separated by \cs{and}.
%    \begin{macrocode}
\define@key{kulemt}{assistant}{\def\kulemt@assistant{#1}}
%    \end{macrocode}
% \begin{macro}{\kulemt@assistant}
% The \meta{assistants} is saved in \cs{kulemt@assistant}. Since the
% |assistant| option is a required option, the command is initialized with
% an error message.
%    \begin{macrocode}
\def\kulemt@assistant{\kulemt@error{No assistant given}}
%    \end{macrocode}
% \end{macro}
%
% \DefineOptions{acyear}{^^A
%   acyear=\meta{acyear}}{default the current academic year}
% This option sets the academic year of the thesis. The \meta{acyear}
% should have a format like ``|{2009|~|--|~|2010}|''. This option should
% probably not be used because the default works quite well.
%    \begin{macrocode}
\define@key{kulemt}{acyear}{\def\kulemt@acyear{#1}}
%    \end{macrocode}
% \begin{macro}{\kulemt@acyear}
% The \meta{acyear} is saved in \cs{kulemt@acyear}. If the value is empty,
% the current academic year must be used.
%    \begin{macrocode}
\def\kulemt@acyear{}
%    \end{macrocode}
% \end{macro}
% \begin{macro}{\kulemt@acyear@t}
% The command \cs{kulemt@acyear@t} typesets the academic year. If
% \cs{kulemt@acyear} hasn't been set yet, its value is computed and stored
% in that command. To allow for the thesis to be printed in September, we
% start the academic year on October 1.
%    \begin{macrocode}
\def\kulemt@acyear@t{%
  \ifx\kulemt@acyear\@empty
    \@tempcnta\year \ifnum\month<10 \advance\@tempcnta\m@ne \fi
    \@tempcntb\@tempcnta \advance\@tempcntb\@ne
    \xdef\kulemt@acyear{\the\@tempcnta\space -- \the\@tempcntb}%
  \fi
  \kulemt@acyear}
%    \end{macrocode}
% \end{macro}
%
% \subsubsubsection{Additional information for the filing card}
% Since not every master requires the use of a filing card, the options
% below are only required when the filing card is used.
%
% \DefineOptions{translatedtitle}{^^A
%   translatedtitle=\meta{title2}}{required option}
% This option specifies the title in the language other than the text language.
%    \begin{macrocode}
\define@key{kulemt}{translatedtitle}{\def\kulemt@translatedtitle{#1}}
%    \end{macrocode}
% \begin{macro}{\kulemt@translatedtitle}
% The \meta{title2} is saved in \cs{kulemt@translatedtitle}. Since the
% option is a required option if the filing card is used, the command is
% initialized with an error message.
%    \begin{macrocode}
\def\kulemt@translatedtitle{%
  \kulemt@error{No translated title available}}
%    \end{macrocode}
% \end{macro}
%
% \DefineOptions{shortabstract}{shortabstract=\meta{short abstract}}{}
% This option specifies the short abstract for the filing card.
%    \begin{macrocode}
\define@key{kulemt}{shortabstract}{\def\kulemt@shortabstract{#1}}
%    \end{macrocode}
% \begin{macro}{\kulemt@shortabstract}
% The \meta{short abstract} is saved in \cs{kulemt@shortabstract}. Since the
% option is a required option if the filing card is used, the command is
% initialized with an error message.
%    \begin{macrocode}
\def\kulemt@shortabstract{%
  \kulemt@error{No short abstract available}}
%    \end{macrocode}
% \end{macro}
%
% \DefineOptions{udc}{udc=\meta{UDC nr}}{}
% This option specifies the UDC number. No UDC number formatting is checked.
%    \begin{macrocode}
\define@key{kulemt}{udc}{\def\kulemt@udc{#1}}
%    \end{macrocode}
% \begin{macro}{\kulemt@udc}
% The \meta{UDC nr} is saved in \cs{kulemt@udc}. Since the option is a
% required option if the filing card is used, the command is initialized
% with an error message.
%    \begin{macrocode}
\def\kulemt@udc{%
  \kulemt@error{UDC number missing}}
%    \end{macrocode}
% \end{macro}
%
% \DefineOptions{keywords}{keywords=\meta{keywordlist}}{}
% This option specifies the list of keywords.
%    \begin{macrocode}
\define@key{kulemt}{keywords}{\def\kulemt@keywords{#1}}
%    \end{macrocode}
% \begin{macro}{\kulemt@keywords}
% The \meta{keywordlist} is saved in \cs{kulemt@keywords}.
%    \begin{macrocode}
\def\kulemt@keywords{}
%    \end{macrocode}
% \end{macro}
%
% \DefineOptions{articletitle}{articletitle=\meta{arttitle}}{}
% This option specifies the title of the thesis article, which is required
% by some masters, to be put on the filing card.
%    \begin{macrocode}
\define@key{kulemt}{articletitle}{\def\kulemt@article@title{#1}}
%    \end{macrocode}
% \begin{macro}{\kulemt@article@title}
% The \meta{arttitle} is saved in \cs{kulemt@article@title}.
%    \begin{macrocode}
\def\kulemt@article@title{}
%    \end{macrocode}
% \end{macro}
%
% \subsubsubsection{Conditionally generating pages}
% \DefineOptions{coverpageonly}{coverpageonly}{}
% When this option is used, only the cover page is generated. If it is not
% used, no cover page is generated.
%    \begin{macrocode}
\kulemt@keynovalue{coverpageonly}{\kulemt@coverpagetrue}
%    \end{macrocode}
% \begin{macro}{\ifkulemt@coverpage}
% The switch |kulemt@coverpage| remembers whether the cover page should be
% generated or not.
%    \begin{macrocode}
\newif\ifkulemt@coverpage
%    \end{macrocode}
% \end{macro}
%
% \DefineOptions{frontpagesonly}{frontpagesonly}{}
% When this option is used, only the front pages (title page, copyright
% page and filing card) are generated. If it is not used, the complete
% document is generated.
%    \begin{macrocode}
\kulemt@keynovalue{frontpagesonly}{\kulemt@frontpagestrue}
%    \end{macrocode}
% \begin{macro}{\ifkulemt@frontpages}
% The switch |kulemt@frontpages| remembers whether only the front pages
% should be generated or not.
%    \begin{macrocode}
\newif\ifkulemt@frontpages
%    \end{macrocode}
% \end{macro}
%
% \DefineOptions{filingcard}{filingcard}{}
% When this option is used, the filing card is printed, even if the master
% doesn't require it. When the master requires a filing card, it will be
% printed anyway regardless of the use of this option.
%    \begin{macrocode}
\kulemt@keynovalue{filingcard}{\kulemt@filingcardtrue}
%    \end{macrocode}
% \begin{macro}{\ifkulemt@filingcard}
% The switch |kulemt@filingcard| tells us whether a filing card should be
% printed or not. Its default is set by \cs{kulemt@set@master}. The option
% |filingcard| makes it true.
%    \begin{macrocode}
\newif\ifkulemt@filingcard
%    \end{macrocode}
% \end{macro}
%
% \subsubsubsection{Other options}
% \DefineOptions{bindcover}{bindcover=\meta{dimen}}{default \texttt{0pt}}
% \changes{v1.6}{2012/05/13}{New option \texttt{bindcover}}
% \changes{v1.7}{2013/04/12}{Option \texttt{bindcover} is obsolete}
% This option specifies the loss \meta{dimen} of visible paper due on the
% cover page only due to the binding tape. Because of the new cover page
% layout, it became obsolete in version~1.7.
%    \begin{macrocode}
\define@key{kulemt}{bindcover}{\ClassWarningNoLine\kulemt@cls{%
    Option `bindcover' is no longer supported}}
%    \end{macrocode}
%
% \subsubsection{Commands for the configuration file}
% \subsubsubsection{Auxiliary commands}
% \begin{macro}{\kulemt@def@master}
% The \cs{kulemt@def@master}\marg{id}\marg{deflist} command defines the
% master specific data for master with abbreviation \meta{id}. The
% \meta{deflist} contains an unseparated list of groups (or single letters)
% with the following meaning:
% \begin{enumerate}
% \item |N| (Dutch) or |E| (English): the master language (the official
%   language of the master)
% \item Number for faculty identification (use braces if more than one
%   digit). See \cs{kulemt@facnum} for more information.
% \item |F| or |N|: always print a filing card (|F|) or not (|N|)
% \item Master colors, using the format
%   ``|{|\meta{background}|:|\meta{text}|}|'' or
%   ``|{|\meta{background}|}|''. The default \meta{background} color is
%   white and the default \meta{text} color is black. When specified, each
%   color is given as a comma separated list of C,M,Y,K fractions.
% \item Master title (between braces)
% \item Optional copyright contact info
%   |{|\meta{address}|:|\meta{phone}|:|\meta{email}|}|. If this element
%   isn't present, the faculty information is used.
% \item Optional unseparated list of master options. Each option is
%   surrounded by braces and consists of an abbreviation, followed by |:|
%   and the title of the option.
% \item Optional unseparated list of obsolete master options. Each option is
%   surrounded by braces and consists of an abbreviation, followed by |:|
%   and the title of the option. You have to make sure that the
%   abbreviation doesn't conflict with an abbreviation of a master option.
% \end{enumerate}
% As soon as an optional element isn't used, any of the following elements
% can't be used either.
%    \begin{macrocode}
\def\kulemt@def@master#1{\@namedef{kulemt@m@#1}}
%    \end{macrocode}
% \end{macro}
% \begin{macro}{\kulemt@obsolete@master}
% \changes{v1.4}{2011/06/07}{New command}^^A
% The \cs{kulemt@obsolete@master}\oarg{spec}\marg{id}\marg{deflist} command
% defines the master specific data in the same way as
% \cs{kulemt@def@master}, but for an obsolete master with abbreviation
% \meta{id}. The optional argument \meta{spec} lets you specify the version
% of the obsolete master. Using the year the master became obsolete, may be
% a good \meta{spec}.
%    \begin{macrocode}
\newcommand*\kulemt@obsolete@master[2][]{%
  \@namedef{kulemt@m@#2\if &#1&\else .#1\fi}}
%    \end{macrocode}
% This way a user can still generate a thesis for an obsolete master by
% using a master id ``\meta{id}|.|\meta{spec}'' or ``\meta{id}'' if
% \meta{spec} is empty or missing.
% \end{macro}
%
% \begin{macro}{\kulemt@set@master}
% \changes{v1.6}{2012/05/13}{Extra argument for obsolete master options.}
% \begin{macro}{\kulemt@master@language}
% \begin{macro}{\kulemt@master@colors}
% \begin{macro}{\kulemt@master@title}
% \begin{macro}{\kulemt@copyright@contact}
% \begin{macro}{\kulemt@master@options}
% The command \cs{kulemt@set@master} isn't a command for the configuration
% file, but it is defined here because it is related to the previous
% command. It sets the commands \cs{kulemt@master@language} (the master
% language), \cs{kulemt@facnum} (the faculty number of the master),
% \cs{kulemt@master@colors} (the master colors formatted as
% ``\meta{bg}\texttt{:}\meta{fg}'' or  ``\meta{bg}''),
% \cs{kulemt@master@title} (the name of the master),
% \cs{kulemt@copyright@contact} (the contact information for the
% copyright), and \cs{kulemt@master@options} (a list of master options).
%    \begin{macrocode}
\def\kulemt@set@master#1#2#3#4#5#6#7#8#9\@nil{%
  \edef\kulemt@master@language{%
    \if N\@car#1\@nil dutch\else english\fi}%
  \kulemt@facnum=#2\relax
  \if F\@car#3\@nil \kulemt@filingcardtrue \fi
  \def\kulemt@master@colors{#4}%
  \def\kulemt@master@title{#5}%
  \def\kulemt@copyright@contact{#6}%
  \ifx\kulemt@copyright@contact\@empty
    \protected@edef\kulemt@copyright@contact{\kulemt@fac@copyright}%
  \fi
  \@ifnextchar[\kulemt@set@mo{\kulemt@set@mo[]}#7\@nil{#8}}
%    \end{macrocode}
% \begin{macro}{\kulemt@set@mo}
% \changes{v1.7}{2013/04/12}{New command.}
% The command \cs{kulemt@set@mo}\oarg{list}\meta{valid}|\@nil|\marg{obsolete}
% stores the information on the master options.
% \begin{macro}{\kulemt@master@options@v}
% \changes{v1.6}{2012/05/13}{New command.}
% \begin{macro}{\kulemt@master@options@o}
% \changes{v1.6}{2012/05/13}{New command.}
% The valid master options \meta{valid} can be found in
% \cs{kulemt@master@options@v}, while the obsolete options \meta{obsolete}
% are stored in \cs{kulemt@master@options@o}.
% \begin{macro}{\kulemt@master@options@vl}
% \changes{v1.7}{2013/04/12}{New command.}
% If setting the master option is required by the master,
% \cs{kulemt@master@options@vl} contains \meta{list}, a list of abbreviated
% master options to choose from.
% If the \meta{list} is ``|-|'', the command is undefined and the option
% variables are emptied.
%    \begin{macrocode}
\def\kulemt@set@mo[#1]#2\@nil#3{%
  \def\kulemt@master@options@vl{#1}%
  \def\kulemt@master@options@v{-}%
  \ifx\kulemt@master@options@vl\kulemt@master@options@v
    \let\kulemt@master@options@vl\relax
    \let\kulemt@master@options@v\@empty
    \let\kulemt@master@options@o\@empty
    \let\kulemt@master@options\@empty
  \else
    \def\kulemt@master@options@v{#2}%
    \def\kulemt@master@options@o{#3}%
    \def\kulemt@master@options{#2#3}%
  \fi}
%    \end{macrocode}
% \end{macro}
% \end{macro}
% \end{macro}
% \end{macro}
% \end{macro}
% \end{macro}
% \end{macro}
% \end{macro}
% \end{macro}
% \end{macro}
%
% \begin{macro}{\kulemt@getcolors}
% \begin{macro}{\kulemt@color@bg}
% \begin{macro}{\kulemt@color@fg}
% The command \cs{kulemt@getcolors} splits the master color specification,
% as specified for \cs{kulemt@master@colors}, into the background and the
% text color. The are stored respectively in \cs{kulemt@color@bg} and
% \cs{kulemt@color@fg}.
%    \begin{macrocode}
\def\kulemt@getcolors#1:#2:#3\@nil{%
  \edef\kulemt@color@bg{\if !#1!{white}\else [cmyk]{#1}\fi}%
  \edef\kulemt@color@fg{\if !#2!{black}\else [cmyk]{#2}\fi}}
%    \end{macrocode}
% \end{macro}
% \end{macro}
% \end{macro}
%
% \begin{macro}{\kulemt@div@master}
% ^^A \changes{v0.95}{2010/01/19}{added}
% When typesetting the contents of the configuration file, it's nice to
% divide up the long list of masters. Therefore the command
% \cs{kulemt@div@master}\marg{head} is introduced. It normally simply
% gobbles its argument \meta{head}. But the user can redefine it before
% inputting |kulemt.cfg| to have different actions on different sections.
%    \begin{macrocode}
\let\kulemt@div@master\@gobble
%    \end{macrocode}
% \end{macro}
%
% \begin{macro}{\kulemt@end@master@def}
% ^^A \changes{v0.99}{2010/01/27}{added}
% The command \cs{kulemt@end@master@def} indicates the end of the master
% information in the configuration file. The only commands allowed before
% it in the configuration file are \cs{ProvidesFile},
% \cs{kulemt@div@master}, and \cs{kulemt@def@master}. All other
% configuration commands in the main configuration file |kulemt.cfg|
% must be placed after this command.
%    \begin{macrocode}
\let\kulemt@end@master@def\relax
%    \end{macrocode}
% \end{macro}
%
% \subsubsubsection{Commands providing defaults}\label{sec:comdef}
% These commands store the default data if no configuration file is used.
% If you define them in the configuration file, you must use
% \cs{renewcommand} (or \cs{def}).
% \begin{macro}{\kulemt@facnum}
% The \cs{kulemt@facnum} count register stores information about which
% faculties are involved. It is used in commands to select faculty
% dependent data. The default value 1 corresponds to the Faculty of
% Engineering Science. The value 0 is used when multiple faculties are involved.
% Values larger than 1 are currently not used so they are available for
% other faculties or combinations.
%    \begin{macrocode}
\newcount\kulemt@facnum
\kulemt@facnum\@ne
%    \end{macrocode}
% \end{macro}
%
% \begin{macro}{\kulemt@fac@name}
% The \cs{kulemt@fac@name} command is used to typeset the faculty name.
% This default implementation only typesets something for
% \cs{kulemt@facnum} equal to one.
%    \begin{macrocode}
\def\kulemt@fac@name{%
  \ifcase\kulemt@facnum \or
    Facult\kulemt@ifdutch{eit Ingenieurswetenschappen}%
                         {y of Engineering}%
  \fi}
%    \end{macrocode}
% \end{macro}
%
% \begin{macro}{\kulemt@kulfac@logo}
% \changes{v1.7}{2013/04/12}{New command}
% The \cs{kulemt@kulfac@logo}\marg{igopts} command is used to typeset the
% KU~Leuven logo, eventually combined with a faculty logo. The
% \meta{igopts} options are passed to a \cs{includegraphics} command.
% This default implementation only typesets the KU~Leuven logo |logokul|
% (|.eps| or |.pdf|). If \cs{kulemt@facnum} equals to one, it typesets the
% combined logo |logokuleng| (|.eps| or |.pdf|).
%    \begin{macrocode}
\def\kulemt@kulfac@logo#1{%
  \edef\@tempa{[#1]{logokul\ifnum\kulemt@facnum=\@ne eng\fi}}%
  \expandafter\includegraphics\@tempa}
%    \end{macrocode}
% \end{macro}
%
% \begin{macro}{\kulemt@fac@logo}
% \changes{v1.7}{2013/04/12}{No-op because a combined logo is used}
% The \cs{kulemt@fac@logo}\marg{igopts} command is used to typeset the
% faculty logo. The \meta{igopts} options are passed to an
% \cs{includegraphics} command. The default implementation is a no-op.
%    \begin{macrocode}
\def\kulemt@fac@logo#1{}
%    \end{macrocode}
% \end{macro}
%
% \begin{macro}{\kulemt@fac@copyright}
% The \cs{kulemt@fac@copyright} command is used to generates the default
% copyright contact information\footnote{See \cs{kulemt@contact@scan} for
%   the format used.}. It must \emph{always} generate contact information!
% This default implementation uses the Faculty of Engineering Science data,
% also if the master belongs to multiple faculties. Other cases
% (\cs{kulemt@facnum}${}>1$) simply refer to the promoter.
%    \begin{macrocode}
\def\kulemt@fac@copyright{%
  \ifnum\kulemt@facnum >\@ne
    \kulemt@ifdutch d{th}e \MakeLowercase{\noexpand\kulemt@paa@0}::%
  \else
    Faculteit Ingenieurswetenschappen, Kasteelpark Arenberg 1 bus 2200,
    B-3001 Heverlee:+32-16-321350:%
  \fi}
%    \end{macrocode}
% Note that this command is used in a moving argument, so only robust
% commands are allowed inside its definition! However we use \cs{noexpand}
% instead of \cs{protect} before \cs{kulemt@paa@} to guard only against the
% first expansion in \cs{kulemt@set@master}. Otherwise \cs{MakeLowercase}
% won't work.
% \end{macro}
%
% \begin{macro}{\kulemt@paa@}
% ^^A \changes{v0.99}{2010/02/26}{Solve bug 1: use a numeric argument to select
% ^^A   the type of heading}
% ^^A \changes{v1.0}{2010/03/02}{Use ``assessoren'' in Dutch}
% The command \cs{kulemt@paa@}\marg{num} generates the labels for promoters
% (if \meta{num} is |0|), assessors (if \meta{num} is |1|), or assistants
% (if \meta{num} is |2|).
%    \begin{macrocode}
\def\kulemt@paa@#1{%
  \ifcase #1%
    Promot\kulemt@ifand\kulemt@promotor{\kulemt@ifdutch{oren}{ers}}{}%
  \or
    Assessor\kulemt@ifand\kulemt@assessor{\kulemt@ifdutch{en}s}{}%
  \or
    \kulemt@ifdutch{Begeleider}{Assistant}%
    \kulemt@ifand\kulemt@assistant{s}{}%
  \fi}
%    \end{macrocode}
% \end{macro}
%
% \subsubsection{Input the configuration file}
% \begin{macro}{\kulemt@cfgfile}\label{def:cfgfile}
% Provide the name of the configuration if the calling class hasn't set it
% yet.
%    \begin{macrocode}
\providecommand*\kulemt@cfgfile{kulemt.cfg}
%    \end{macrocode}
% \end{macro}
% Input the file if it exists.
%    \begin{macrocode}
\ifx\kulemt@cfgfile\@empty\else
  \InputIfFileExists\kulemt@cfgfile{}{\kulemt@error{%
      Configuration file `\kulemt@cfgfile' is not installed}}
\fi
%    \end{macrocode}
%
% \subsubsection{Process the class options}
%    \begin{macrocode}
\kulemt@process@ptions
%    \end{macrocode}
% Then we process the required |master| option, which must exist and must
% be defined in the configuration file.
%    \begin{macrocode}
\ifx\kulemt@opt@master\@empty
  \kulemt@error{Required option `master' missing}\fi
\expandafter\let\expandafter\@tempa
  \csname kulemt@m@\kulemt@opt@master \endcsname
\ifx\@tempa\relax
  \kulemt@error{%
    Unsupported value `\kulemt@opt@master' for option `master'}\fi
\expandafter\kulemt@set@master\@tempa{}{}{}\@nil
%    \end{macrocode}
%
% ^^A \changes{v0.99}{2010/02/27}{Additional master specific configuration file
% ^^A   is loaded}
% \begin{macro}{\kulemt@cfgfile@m}
% Provide the name of the master specific configuration file if the calling
% class hasn't set it yet.
%    \begin{macrocode}
\@ifundefined{kulemt@cfgfile@m}{%
  \edef\kulemt@cfgfile@m{kulemt-\kulemt@opt@master.cfg}}{}
%    \end{macrocode}
% \end{macro}
% Input the file if it exists.
%    \begin{macrocode}
\ifx\kulemt@cfgfile@m\@empty\else
  \InputIfFileExists\kulemt@cfgfile@m{}{}
\fi
%    \end{macrocode}
%
% \begin{macro}{\kulemt@check@masteroption}
% Once we know the master, we can check the master option and eventually
% expand the abbreviation. Since the master option can be set with
% \cs{setup}, we have to check this at the end of the document preamble,
% but only if the master defines options.
% \changes{v1.6}{2012/05/13}{Missing master options generate a warning
%   instead of an error.}^^A
% \changes{v1.7}{2013/04/12}{Generate an error again, but only when the
%   master requires options explicitly.}^^A
% An error is raised if the master defines a valid option list and no
% master option is given.
%    \begin{macrocode}
\ifx\kulemt@master@options@vl\@empty\else
  \ifx\kulemt@master@options@vl\relax\else
    \def\kulemt@check@masteroption{%
      \ifx\kulemt@master@option\@empty
        \kulemt@error{%
          When using option `master=\kulemt@opt@master',\MessageBreak
          you should specify at least one master option.\MessageBreak
          Allowed master options are: \kulemt@master@options@vl}%
      \fi}
    \AtBeginDocument{\kulemt@check@masteroption}
  \fi
\fi
%    \end{macrocode}
% \end{macro}
%
% \subsection{Loading the required class and packages}
% \subsubsection{The \cls{memoir} class}
% This class is based on \cls{memoir} using A4 paper. Most of its
% parameters are set later on in the document layout section on
% page~\pageref{sec:doclayout}.
%    \begin{macrocode}
\LoadClass[a4paper,1\kulemt@ptsize pt]{memoir}[2004/04/05]
%    \end{macrocode}
% \begin{macro}{\and}
% For the \pkg{hyperref} option |pdfusetitle| to work correctly, we redefine
% \cs{and} as a newline.
%    \begin{macrocode}
\def\and{\\}
%    \end{macrocode}
% \end{macro}
% \begin{macro}{\andnext}
% At the same time the \cls{memoir} command \cs{andnext} gets the same
% definition.
%    \begin{macrocode}
\let\andnext\and
%    \end{macrocode}
% \end{macro}
% 
% \subsubsection{The \pkg{babel} package}
% We use the \pkg{babel} options stored in \cs{kulemt@babel@opt}.
%    \begin{macrocode}
\RequirePackage[\kulemt@babel@opt]{babel}
%    \end{macrocode}
% English and Dutch translations of additional \cls{memoir} commands are
% also provided.
%    \begin{macrocode}
\addto\captionsenglish{%
  \def\appendixtocname{Appendices}%
  \def\appendixpagename{Appendices}%
  \def\figurerefname{Figure}%
  \def\tablerefname{Table}%
  \def\pagerefname{page}%
  \def\partrefname{Part~}%
  \def\chapterrefname{Chapter~}%
  \def\listfiguresandtablesname{List of Figures and Tables}}
\begingroup
  \catcode`\"\active
  \@firstofone{\endgroup
    \addto\captionsdutch{%
      \def\appendixtocname{B"ylagen}%
      \def\appendixpagename{B"ylagen}%
      \def\figurerefname{figuur}%
      \def\tablerefname{tabel}%
      \def\pagerefname{pagina}%
      \def\partrefname{Deel~}%
      \def\chapterrefname{hoofdstuk~}%
      \def\listfiguresandtablesname{L"yst van figuren en tabellen}}}
%    \end{macrocode}
% \begin{macro}{\latinencoding}
% \changes{v1.5}{2011/08/10}{Set to T1 only if T1 is the current encoding.}
% The T1 font encoding is always defined for the front page (cf.\
% page~\pageref{t1def}). This confuses \pkg{babel}. So we redefine
% \cs{latinencoding} to look only at the current font encoding and ignore
% the fact that the T1 encoding is loaded. This resolves a \pkg{microtype}
% error when using the default Computer Modern fonts.
%    \begin{macrocode}
\AtBeginDocument{\gdef\latinencoding{T1}%
  \ifx\cf@encoding\latinencoding\else \xdef\latinencoding{OT1}\fi}
%    \end{macrocode}
% \end{macro}
%
% Finally the main language is set to the text language. Since
% \cs{main@language} must be fully expanded, \cs{kulemt@language} is
% expanded first. This guarantees that \cs{kulemt@language} can be used
% later on directly as a \pkg{babel} language.\\
% Note: active characters are only activated after |\begin{document}|.
%    \begin{macrocode}
\edef\kulemt@language{\kulemt@language}
\expandafter\main@language\expandafter{\kulemt@language}
%    \end{macrocode}
%
% \begin{macro}{\kulemt@selectmasterlanguage}
% The shorthand command \cs{kulemt@selectmasterlanguage} switches to the
% official master language.
%    \begin{macrocode}
\def\kulemt@selectmasterlanguage{%
  \expandafter\selectlanguage\expandafter{\kulemt@master@language}}
%    \end{macrocode}
% \end{macro}
%
% \begin{macro}{\kulemt@selecttextlanguage}
% The shorthand command \cs{kulemt@selecttextlanguage} switches to the main
% text language.
%    \begin{macrocode}
\def\kulemt@selecttextlanguage{%
  \expandafter\selectlanguage\expandafter{\kulemt@language}}
%    \end{macrocode}
% \end{macro}
% 
% \subsubsection{The \pkg{graphicx} and \pkg{color} package}
% The package \pkg{graphicx} is needed for including images on the cover
% and the title page.
%    \begin{macrocode}
\RequirePackage{graphicx}
%    \end{macrocode}
% The package \pkg{color} is needed for the cover page, but it is also used
% to color the hyperlinks.
%    \begin{macrocode}
\RequirePackage{color}
%    \end{macrocode}
% 
% \subsubsection{The \pkg{microtype} package}
% Using the \pkg{microtype} package results in much nicer output: less
% overfull hboxes and less hyphenation. The user can always setup or
% disable \pkg{microtype} with \cs{microtypesetup}.
% \begin{macro}{\kulemt@loadmt}
% Older versions must be loaded after font definitions, so we postpone
% requiring the package with \cs{kulemt@loadmt}. The best place to load it
% is after the font declaration, so the user can put a \cs{microtypesetup}
% after it. Therefore it will be loaded after the \cs{setup} which declares
% the fonts (see p.~\pageref{loadmt-after-font}). If the user doesn't use
% the |font| option, it must be loaded at the end of the preamble. This
% implies that \cs{kulemt@loadmt} can be called twice. We also have to take
% into account that the user herself may have required the package already
% in the preamble, e.g., with options.
%    \begin{macrocode}
\def\kulemt@loadmt{%
  \@ifpackageloaded{microtype}{}{\RequirePackage{microtype}}}
\AtBeginDocument{\kulemt@loadmt}
%    \end{macrocode}
% \end{macro}
% The package \pkg{microtype} is not available in older installation, so
% it's only used when available and wanted by the user (option
% |nomicrotype|) and pdfTeX is used in PDF mode. When the package is not
% used, a message is put in the log file.\\
% In the following code, \cs{@tempa} temporarily stores the reason why the
% package wasn't loaded.
%    \begin{macrocode}
\ifkulemt@microtype
  \ifpdf
    \IfFileExists{microtype.sty}{}{%
      \def\@tempa{the package is not installed}%
      \kulemt@microtypefalse}
  \else
    \def\@tempa{you're not using pdflatex in pdf mode}
    \kulemt@microtypefalse
  \fi
\else
  \def\@tempa{option `nomicrotype' was used}
\fi
\ifkulemt@microtype\else
  \let\kulemt@loadmt\relax
  \ClassWarningNoLine\kulemt@cls{%
    Package `microtype' not used because\MessageBreak
    \@tempa}%
\fi
%    \end{macrocode}
% 
% \subsubsection{The \pkg{hyperref} package}
% ^^A \changes{v0.93}{2010/01/15}{Package \textsf{hyperref} no longer loaded}
% The package \pkg{hyperref} is wanted to create useful PDF files. Because
% it interacts with many other packages, it is not loaded by default.
%
% \begin{macro}{\kulemt@check@hyperref}
% If \pkg{hyperref} has been been loaded, some additional actions are
% needed, which are stored in \cs{kulemt@check@hyperref}.
%    \begin{macrocode}
\def\kulemt@check@hyperref{%
  \@ifpackageloaded{hyperref}{%
%    \end{macrocode}
% \begin{macro}{\HyPsd@babel@dutch}
% It seems that some \pkg{babel} ligatures are not translated to an
% equivalent character sequence for the bookmarks. I guess it should be
% reported as a feature request, but for the time being and for older
% versions, we provide them ourselves. 
%    \begin{macrocode}
    \@ifundefined{HyPsd@babel@dutch}{}{%
      \addto\HyPsd@babel@dutch{%
        \declare@shorthand{dutch}{"y}{ij}%
        \declare@shorthand{dutch}{"Y}{IJ}}}%
%    \end{macrocode}
% \end{macro}
% The package \pkg{memhfixc} provides \pkg{hyperref} related fixes and
% extensions for \pkg{memoir}. Newer versions of \pkg{hyperref} load this
% automatically, but we require it for older versions.
%    \begin{macrocode}
    \@ifpackageloaded{memhfixc}{}{%
      \RequirePackage{memhfixc}}%
%    \end{macrocode}
% \begin{macro}{\theHsubfigure}
% ^^A \changes{v0.96}{2010/01/24}{provided}
% \begin{macro}{\theHsubtable}
% ^^A \changes{v0.96}{2010/01/24}{provided}
% To avoid name conflicts, subfloats should be numbered within the parent
% float. The defaults are provided for the most common cases of subfigures
% and subtables.
%    \begin{macrocode}
    \providecommand*\theHsubfigure{\theHfigure.\arabic{subfigure}}%
    \providecommand*\theHsubtable{\theHtable.\arabic{subtable}}%
  }{}}
%    \end{macrocode}
% \end{macro}
% \end{macro}
% \end{macro}
% The actions from \cs{kulemt@check@hyperref} are executed after all
% packages are loaded.
%    \begin{macrocode}
\AtBeginDocument{\kulemt@check@hyperref}
%    \end{macrocode}
%
% \subsubsection*{End of option handling}
% Now is the time to check the one time options for the first time. They
% will be checked again after each \cs{setup}.\\
% But before loading fonts, we make sure the T1\label{t1def} encoding is
% defined for the front page font (cf.\ \S\ref{sec:frontpagefont}) and the
% default \LaTeX\ font encoding OT1 is selected again.
%    \begin{macrocode}
\RequirePackage[T1,OT1]{fontenc}
\kulemt@do@once@opts
%    \end{macrocode}
%
% \subsection{Document layout}\label{sec:doclayout}
% \subsubsection{Page layout}
% The default \cs{headheight} and \cs{headsep} from \pkg{memoir} are left
% as is, but the text body dimensions are redefined depending on the text
% point size (10pt and 11pt respectively).
%    \begin{macrocode}
\ifcase\kulemt@ptsize\relax
  \textwidth=13cm
  \textheight=20cm
\or
  \textwidth=14cm
  \textheight=215mm
\fi
%    \end{macrocode}
% The inner (\cs{spinemargin}) and outer (\cs{foremargin}) margins are
% computed as follows:\\
% \hspace*{2em}$\begin{array}[t]{@{}l@{{}={}}l@{}}
%   \cs{foremargin}  & 0.6\,(\cs{paperwidth} - \cs{textwidth} -
%                            \mathrm{binding}) \\
%   \cs{spinemargin} & 0.4\,(\cs{paperwidth} - \cs{textwidth} -
%                            \mathrm{binding}) + \mathrm{binding} \\
% \end{array}$ \\
% For one side layout, the visible parts of both margins are made equal
% (use a factor 0.5 instead of 0.4 and 0.6).
%    \begin{macrocode}
\spinemargin\paperwidth
\advance\spinemargin -\textwidth
\foremargin\spinemargin
\advance\foremargin -\kulemt@bind\relax
\foremargin .\if@twoside 6\else 5\fi\foremargin
\advance\spinemargin -\foremargin
%    \end{macrocode}
% Margin notes get a fixed width independent of one side or two side
% printing. This makes sure that printing that the PDF distribution (one
% side) and the printed version (two side) have the same text on each page.
% The separation between notes is kept as small as possible, as well as the
% distance from the text block.
%    \begin{macrocode}
\marginparwidth=56pt
\marginparsep=1.2\onelineskip
\marginparpush=\onelineskip
%    \end{macrocode}
% The lower margin is 1.2 times the upper margin. The header parameters
% are set to the default values.
%    \begin{macrocode}
\setulmargins{*}{*}{1.2}
\setheaderspaces{*}{\headsep}{*}
%    \end{macrocode}
% Finish up the layout definitions. Redo this at the end of the document
% preamble in case the user redefines some parameters (which she shouldn't
% of course).
%    \begin{macrocode}
\checkthelayout\fixthelayout
\AtBeginDocument{\checkandfixthelayout}
%    \end{macrocode}
%
% \begin{macro}{\clearforchapter}
% ^^A \changes{v0.92}{2010/01/10}{redefinition}
% The |open|\ldots options only control the main matter chapters. Chapters
% in the front and back matter are always |openany|. If you don't like
% this, you can use the \cs{openleft} or \cs{openright} command in the
% document.
%    \begin{macrocode}
\renewcommand*\clearforchapter{%
  \if@mainmatter
    \if@openleft
      \cleartoverso
    \else
      \if@openright
        \cleartorecto
      \else
        \clearpage
      \fi
    \fi
  \else
    \clearpage
  \fi}
%    \end{macrocode}
% \end{macro}
%
% \subsubsection{Page styles}
% \DefinePageStyle{ruled}
% By default the pagestyle \pstyle{ruled} is used. However for front matter
% (actually for non-main matter) the header on odd pages is the same as on
% even pages, because typically front matter chapters have no sections.
%    \begin{macrocode}
\makeoddhead{ruled}{}{}{%
  \if@mainmatter \rightmark \else \scshape\leftmark \fi}
\pagestyle{ruled}
%    \end{macrocode}
%
% ^^A \changes{v0.97}{2010/01/26}{Aliased the `\pstyle{chapter}' pagestyle to
% ^^A   the new `\pstyle{nohead}' pagestyle}
% \DefinePageStyle{nohead}
% The \pstyle{nohead} pagestyle puts the page number in the footer at the
% outer margin.
%    \begin{macrocode}
\makepagestyle{nohead}
\makeevenfoot{nohead}{\thepage}{}{}
\makeoddfoot{nohead}{}{}{\thepage}
%    \end{macrocode}
% \DefinePageStyle{nohead}
% The \pstyle{chapter} pagestyle is aliased to this new pagestyle.
%    \begin{macrocode}
\aliaspagestyle{chapter}{nohead}
%    \end{macrocode}
%
% \subsubsection{Section numbering}
% Sections are numbered up to the subsection level.
%    \begin{macrocode}
\maxsecnumdepth{subsection}
%    \end{macrocode}
% But numbering in the table of contents ends at the section level.
%    \begin{macrocode}
\maxtocdepth{section}
%    \end{macrocode}
%
% \subsubsection{Content lists}
% In \cls{memoir}, content lists don't start a new page. By default it is
% done here. But in case only a few figures and tables are used, the new
% \cs{listoffiguresandtables} can be used.
%    \begin{macrocode}
\def\tocheadstart{\clearforchapter\chapterheadstart}
\def\lofheadstart{\clearforchapter\chapterheadstart}
\def\lotheadstart{\clearforchapter\chapterheadstart}
%    \end{macrocode}
% \begin{macro}{\listoffiguresandtables}
% The command \cs{listoffiguresandtables} list first the figures and then
% the tables on the same page.
%    \begin{macrocode}
\newcommand*\listoffiguresandtables{%
  \chapter\listfiguresandtablesname
  \def\@lofmaketitle{\section*\listfigurename}%
  \listoffigures*%
  \let\listoffigures\relax
  \def\@lotmaketitle{\section*\listtablename}%
  \listoftables*%
  \let\listoftables\relax}
%    \end{macrocode}
% \end{macro}
% \begin{macro}{\listfiguresandtablesname}
% The command \cs{listfiguresandtablesname} generates the title for a page
% combining the list of figures and of tables.
%    \begin{macrocode}
\newcommand*\listfiguresandtablesname{List of Figures and Tables}
%    \end{macrocode}
% \end{macro}
% The content lists are typeset ragged right without hyphenation.
%    \begin{macrocode}
\setrmarg{2.55em plus1fil}
%    \end{macrocode}
% For these lists the space before chapter items is halved.
%    \begin{macrocode}
\setlength{\cftbeforechapterskip}{1ex \@plus\p@}
%    \end{macrocode}
%
% \subsubsection{Tables and figures}
% The captions of tables and figures have the last line centered. The
% caption name is printed in small caps. Because of bugs in some versions
% of \cls{memoir} the font settings for the caption name must be undone for
% the caption title.
%    \begin{macrocode}
\captionnamefont{\scshape}
\captiontitlefont{\upshape}
\captionstyle[\centering]{\centerlastline}
%    \end{macrocode}
%
% \subsection{Front material}
% The front material consists of the cover page, the title page, the
% copyright page, and the filing card. Since either the cover page or the
% title page are printed, we call both the front page.
% \subsubsection{Front page font}\label{sec:frontpagefont}
% \changes{v1.1}{2010/03/07}{The \texttt{phv} font seems to require a T1
%   encoding to work for accented characters. This also means it can be
%   defined at the class loading time.}
% For the cover page and the title page the Helvetica font must be used. To
% avoid collisions with scaled Helvetica body fonts, a specific front page
% font is defined based on unscaled Helvetica. It seems that we need a T1
% encoding to print accented characters when \pkg{babel} is used for Dutch.\\
% Note: Since the font shapes are defined outside of an |.fd| file, spaces
% are \emph{not} ignored in the definitions. So remove all spaces!
%    \begin{macrocode}
\DeclareFontFamily{T1}{kulemtfpf}{}
\DeclareFontShape{T1}{kulemtfpf}{m}{n}{<->phvr8t}{}
\DeclareFontShape{T1}{kulemtfpf}{m}{sc}{<->phvrc8t}{}
\DeclareFontShape{T1}{kulemtfpf}{m}{sl}{<->phvro8t}{}
\DeclareFontShape{T1}{kulemtfpf}{m}{it}{<->ssub*kulemtfpf/m/sl}{}
\DeclareFontShape{T1}{kulemtfpf}{bx}{n}{<->phvb8t}{}
\DeclareFontShape{T1}{kulemtfpf}{bx}{sc}{<->phvbc8t}{}
\DeclareFontShape{T1}{kulemtfpf}{bx}{sl}{<->phvbo8t}{}
\DeclareFontShape{T1}{kulemtfpf}{bx}{it}{<->ssub*kulemtfpf/bx/it}{}
\DeclareFontShape{T1}{kulemtfpf}{b}{n}{<->ssub*kulemtfpf/bx/n}{}
\DeclareFontShape{T1}{kulemtfpf}{b}{sc}{<->ssub*kulemtfpf/bx/sc}{}
\DeclareFontShape{T1}{kulemtfpf}{b}{sl}{<->ssub*kulemtfpf/bx/sl}{}
\DeclareFontShape{T1}{kulemtfpf}{b}{it}{<->ssub*kulemtfpf/bx/sl}{}
%    \end{macrocode}
%
% ^^A \changes{v0.99}{2010/02/21}{Reduce the font sizes on the cover
% ^^A    and title page}
% \begin{macro}{\kulemt@fpf@title}
% This command selects the font used for the title (24.88\,pt Helvetica).
%    \begin{macrocode}
\def\kulemt@fpf@title{\fontsize\@xxvpt{30}\selectfont}
%    \end{macrocode}
% \end{macro}
% \begin{macro}{\kulemt@fpf@subtitle}
% This command selects the font used for the subtitle (17.28\,pt Helvetica).
%    \begin{macrocode}
\def\kulemt@fpf@subtitle{\fontsize\@xviipt{22}\selectfont}
%    \end{macrocode}
% \end{macro}
% \begin{macro}{\kulemt@fpf@author}
% This command selects the font used for the author (14.4\,pt Helvetica).
%    \begin{macrocode}
\def\kulemt@fpf@author{\fontsize\@xivpt{18}\selectfont}
%    \end{macrocode}
% \end{macro}
% \begin{macro}{\kulemt@fpf@txthead}
% This command selects the font used for the text heading (12\,pt Helvetica
% bold).
%    \begin{macrocode}
\def\kulemt@fpf@txthead{\fontsize\@xiipt{14.5}%
  \fontseries\bfdefault\selectfont}
%    \end{macrocode}
% \end{macro}
% \begin{macro}{\kulemt@fpf@text}
% This command selects the font used for the ordinary text (12\,pt Helvetica).
%    \begin{macrocode}
\def\kulemt@fpf@text{\fontsize\@xiipt{14}\selectfont}
%    \end{macrocode}
% \end{macro}
% \begin{macro}{\kulemt@fpf@banner}
% This command selects the font used for the banner at the bottom of the
% page (14.4\,pt Helvetica bold). The academic year text above the banner
% is printed in the non-bold version of this font.
%    \begin{macrocode}
\def\kulemt@fpf@banner{\fontsize\@xivpt{18}%
  \fontseries\bfdefault\selectfont}
%    \end{macrocode}
% \end{macro}
%
% \subsubsection{Utility commands}
% \begin{macro}{\kulemt@master@text}
% The command \cs{kulemt@master@text} prints the full master degree text
% including the master option or major topic. Multiple master options are
% separated by ``and'' (or ``en'' in Dutch).
%    \begin{macrocode}
\def\kulemt@master@text{Thesis
  \kulemt@ifdutch
    {voorgedragen tot het behalen van de graad van}%
    {submitted for the degree of}
  \kulemt@master@title
  \ifx\kulemt@master@option\@empty\else
    \def\@tempb{, }%
    \@for\@tempa:=\kulemt@master@option\do{%
      \ifx\@tempa\@empty\else
        \@tempb \def\@tempb{ \kulemt@ifdutch{en}{and} }%
        \@tempa
      \fi}%
  \fi}
%    \end{macrocode}
% \end{macro}
%
% \begin{macro}{\kulemt@paa@fp}
% ^^A \changes{v0.99}{2010/02/26}{Solve bug 1: use a numeric argument to select
% ^^A   the type of heading}
% The command \cs{kulemtpaa@fp}\marg{num} typesets the promoters,
% assessors, or assistants in front page format. The possible values of
% \meta{num} are the same as the argument \meta{num} of \cs{kulemt@paa@}.
% ^^A \changes{v0.99}{2010/02/21}{Move bold setting to \cs{kulemt@fpf@txthead}}
% \changes{v1.2}{2010/08/03}{Make it a no-op for an empty keyword value}
%    \begin{macrocode}
\def\kulemt@paa@fp#1{%
  \begingroup
    \ifcase #1\relax
      \let\@tempa\kulemt@promotor
    \or
      \let\@tempa\kulemt@assessor
    \or
      \let\@tempa\kulemt@assistant
    \else
       \let\@tempa\@empty
    \fi
    \ifx\@tempa\@empty\else
      \medskip \begingroup 
        \kulemt@fpf@txthead \kulemt@paa@{#1}:\vskip 2\p@
      \endgroup \@tempa \par
    \fi
  \endgroup}
%    \end{macrocode}
% \end{macro}
%
% \subsubsection{Typesetting the front page}
% The front page is either the cover page or the title page. The
% distinction between them is made by the switch |kulemt@coverpage|: when
% true, the cover page is generated, otherwise the title page.
% \begin{macro}{\kulemt@frontpage}
% The front page contains no header or footer. It start in the front matter
% as page~-1. Since it is followed by the copyright page (page~0), the
% first real page can start at~1.
%    \begin{macrocode}
\def\kulemt@frontpage{\clearpage
  \setcounter{page}\m@ne
  \thispagestyle{empty}%
%    \end{macrocode}
% The text on the front page starts on 1\,cm below the upper page edge.
%    \begin{macrocode}
  \@tempdima\uppermargin \advance\@tempdima\topskip
  \advance\@tempdima\baselineskip \advance\@tempdima -1cm%
  \null \vskip -\@tempdima
%    \end{macrocode}
% The typeset area on the first page is different from the rest of the
% text. It is always centered horizontally, also with two side printing.
% The binding offset is ignored as well.
% \changes{v1.7}{2013/04/12}{New logo handling and positioning}^^A
%% The margins are 16\,mm, resulting in a text width of 178\,mm on A4 paper.
% The margins are 2\,cm, resulting in a text width of 17\,cm on A4 paper.
%    \begin{macrocode}
  \hbox to\hsize{%
    \@tempdima 2cm\advance\@tempdima -\spinemargin \hskip\@tempdima
    \vbox to\z@{\hsize 17cm\relax
%    \end{macrocode}
% All elements on the front page are positioned, so avoid inserting
% automatic glue. 
%    \begin{macrocode}
      \lineskip\z@skip \parskip\z@skip
%    \end{macrocode}
% \changes{v1.7}{2013/04/12}{Disable \pkg{microtype} in the front pages}
% The cover page may be generated by a server at a different location. To
% make sure that the typesetting of the front pages doesn't depend on the
% presence of the \pkg{microtype} package. Therefore micro-typography is
% disabled in a front page.
%    \begin{macrocode}
      \@ifundefined{microtypesetup}{}{\microtypesetup{activate=false}}%
%    \end{macrocode}
% The front page text is typeset ragged right in Helvetica using the master
% language.
%    \begin{macrocode}
      \fontencoding{T1}\fontfamily{kulemtfpf}\kulemt@fpf@text
      \raggedright \kulemt@selectmasterlanguage
%    \end{macrocode}
% The first line contains the KU~Leuven logo on the left, eventually
% combined with a faculty logo, and a separate faculty logo on the right.
% The height of this logo line is 3\,cm, which corresponds to the height of
% the combined KU~Leuven-faculty logo. The KU~Leuven logo (without attached
% faculty logo) has fixed dimensions (56\,mm, 2\,cm). The optional faculty
% logo on the right has the same height.
% The left margin of the KU~Leuven logo is 1\,cm.
%    \begin{macrocode}
      \noindent \hskip-1cm%
      \vbox to3cm{\hbox{\kulemt@kulfac@logo{width=56mm}}\vss}\hfill
      \vbox to3cm{\hbox{\kulemt@fac@logo{height=2cm}}\vss}%
      \hskip-1cm\hskip\z@skip
%    \end{macrocode}
% The last |\hskip\z@skip| is needed because the last skip is always
% removed by the paragraph builder.\\
% The minimal space before the title is 40\,pt but it stretches twice as
% fast as the space below the author.\\
% The title and the subtitle are printed in the main text language.
%    \begin{macrocode}
      \vskip 40\p@ \@plus 2fill\relax
      \begingroup \kulemt@selecttextlanguage
        \kulemt@fpf@title \@title \par
%    \end{macrocode}
% If a subtitle is given, it is typeset at the appropriate size and at a
% fixed distance below the title.
%    \begin{macrocode}
        \ifx\kulemt@subtitle\@empty\else
          \vskip 1em\relax
          \kulemt@fpf@subtitle \kulemt@subtitle \par
        \fi
      \endgroup
%    \end{macrocode}
% The minimal space before the authors is again 40\,pt but with a very
% limited stretching. The space after it is 30\,pt with the standard
% stretching.
%    \begin{macrocode}
      \vskip 40\p@ \@plus .3fill%
      \begingroup \kulemt@fpf@author \@author
        \vskip 30\p@ \@plus 1fill\endgroup
%    \end{macrocode}
% The rest is ordinary text which is typeset ragged left, occupying at most
% half of the text body. First comes the degree, followed by the
% promoter(s). On the title page, the assessors and the assistants are
% also listed. The space below this text is 20\,pt with the same stretching
% as above the title.
%    \begin{macrocode}
      \noindent \hfill \vbox{\hsize .5\textwidth \raggedleft
        \kulemt@master@text \par
        \kulemt@paa@fp0%
        \ifkulemt@coverpage\else
          \kulemt@paa@fp1%
          \kulemt@paa@fp2%
        \fi}%
      \vskip 20\p@ \@plus 2fill\relax
%    \end{macrocode}
% The academic year is printed below the text and centered on the page,
% with a space of 15\,pt below it.
% ^^A \changes{v0.99}{2010/02/21}{Academic year uses the non-bold banner font}
%    \begin{macrocode}
      \centering \kulemt@fpf@banner
      \textmd{Academi\kulemt@ifdutch{ejaar}{c year} \kulemt@acyear@t}%
      \vskip 15\p@
%    \end{macrocode}
% \changes{v1.7}{2013/04/12}{Color banner width set to 19\,cm}
% A 19\,cm wide and 15\,mm high color banner with the master degree name is
% printed on the cover page only, placed 30\,pt below the academic year.
% The banner is only printed if the colors \cs{kulemt@master@colors} are
% defined.
%    \begin{macrocode}
      \ifkulemt@coverpage
        \ifx\kulemt@master@colors\@empty\else
          \vskip 15\p@
          \centerline{\fboxsep\z@
            \expandafter\kulemt@getcolors\kulemt@master@colors::\@nil
            \expandafter\colorbox\kulemt@color@bg{%
              \vbox to 15mm{\hsize 19cm\vss
                \expandafter\textcolor\kulemt@color@fg{%
                  \kulemt@master@title}\vss}}}%
        \fi
      \fi
%    \end{macrocode}
% A bottom margin of 1\,cm results on A4 paper in a body height of 27.7\,cm.
%    \begin{macrocode}
      \vskip -277mm}%
    \hss}%
  \clearpage}
%    \end{macrocode}
% \end{macro}
% \begin{macro}{\maketitle}
% Because the previous command prints the title information, the command
% \cs{maketitle} is undefined to avoid problems.
%    \begin{macrocode}
\let\maketitle\relax
%    \end{macrocode}
% \end{macro}
%
% \subsubsection{Typesetting the copyright page}
% \begin{macro}{\kulemt@contact@print}
% \begin{macro}{\kulemt@contact@scan}
% The command \cs{kulemt@contact@print} prints the copyright contact
% information stored in \cs{kulemt@copyright@contact}. The format of the
% contact information is ``\meta{address}|:|\meta{tel}|:|\meta{email}''.
% The \meta{address} must be written to follow ``addressed to'' in English
% and to follow ``wend u tot'' in Dutch. The \meta{address} must be
% present, the telephone number \meta{tel} and the \meta{email} may be
% missing.
%    \begin{macrocode}
\def\kulemt@contact@print{%
  \expandafter\kulemt@contact@scan\kulemt@copyright@contact:::\@nil}
\def\kulemt@contact@scan#1:#2:#3:#4\@nil{#1%
  \def\@tempa{#2}\def\@tempb{#3}%
  \ifx\@tempa\@empty
    \ifx\@tempb\@empty\else , \texttt{#3}\fi
  \else
    , #2%
    \ifx\@tempb\@empty\else
      \space o\kulemt@ifdutch{f via e-}{r by e}mail \texttt{#3}%
    \fi
  \fi}
%    \end{macrocode}
% \end{macro}
% \end{macro}
% A command \cs{kulemt@copyright@}\meta{lang} must defined for every
% existing master and text language \meta{lang}. It contains the copyright
% text in the language \meta{lang}.
% \begin{macro}{\kulemt@copyright@english}
% The command \cs{kulemt@copyright@english} contains the copyright text in
% English.^^A
% \changes{v1.4}{2011/06/07}{Remove the hardcoded ``promotor''}
%    \begin{macrocode}
\def\kulemt@copyright@english{\selectlanguage{english}%
  Without written permission of the \MakeLowercase{\kulemt@paa@0} and
  the author\kulemt@ifand\@author{s}{} it is forbidden to reproduce
  or adapt in any form or by any means any part of this publication.
  Requests for obtaining the right to reproduce or utilize parts of
  this publication should be addressed to \kulemt@contact@print.\par
  A written permission of the \MakeLowercase{\kulemt@paa@0} is also
  required to use the methods, products, schematics and programs
  described in this work for industrial or commercial use, and for
  submitting this publication in scientific contests.\par}
%    \end{macrocode}
% \end{macro}
% \begin{macro}{\kulemt@copyright@dutch}
% The command \cs{kulemt@copyright@dutch} contains the copyright text in Dutch.
% ^^A \changes{v0.99}{2010/02/21}{Replaced ``wendt U'' by ``wend u''}
% \changes{v1.4}{2011/06/07}{Remove the hardcoded ``promotor''}
%    \begin{macrocode}
\def\kulemt@copyright@dutch{\selectlanguage{dutch}%
  Zonder voorafgaande schriftelijke toestemming van zowel de
  \MakeLowercase{\kulemt@paa@0} als de auteur\kulemt@ifand\@author{s}{}
  is overnemen, kopi\"eren, gebruiken of realiseren van deze uitgave
  of gedeelten ervan verboden. Voor aanvragen tot of informatie
  i.v.m.\ het overnemen en/of gebruik en/of realisatie van gedeelten
  uit deze publicatie, wend u tot \kulemt@contact@print.\par
  Voorafgaande schriftelijke toestemming van de
  \MakeLowercase{\kulemt@paa@0} is eveneens vereist voor het
  aanwenden van de in deze masterproef beschreven (originele)
  methoden, producten, schakelingen en programma's voor industrieel
  of commercieel nut en voor de inzending van deze publicatie ter
  deelname aan wetenschappelijke prijzen of wedstrijden.\par}
%    \end{macrocode}
% \end{macro}
% \begin{macro}{\kulemt@copyrightpage}
% The copyright page contains no header or footer, with the copyright
% notice at the bottom of the page. Paragraphs in the copyright notice are
% typeset without indentation and half a line of spacing between them. To
% avoid hyphenation as much as possible, \cs{sloppypar} is used.
%    \begin{macrocode}
\def\kulemt@copyrightpage{\clearpage
  \thispagestyle{empty}%
  \null \vfill
  \begingroup
    \parindent\z@ \parskip .5\baselineskip \sloppypar
    \copyright\space Copyright KU~Leuven\vskip\baselineskip
%    \end{macrocode}
% ^^A \changes{v0.98}{2010/01/27}{English copyright first}
% If the text and the master language are the same, a copyright notice is
% printed in that language. If  they differ, the English version comes first.\\
% Note: Because of catcode differences we can't compare the master language
% \cs{kulemt@master@language} and the text language \cs{kulemt@language}
% directly.
% \changes{v1.8}{2015/05/25}{Make sure \cs{@tempb} is not overwritten by
%   \cs{@tempa}}
%    \begin{macrocode}
    \expandafter\let\expandafter\@tempa
      \csname kulemt@copyright@\kulemt@master@language\endcsname
    \expandafter\let\expandafter\@tempb
      \csname kulemt@copyright@\kulemt@language\endcsname
    \ifx\@tempa\@tempb \@tempa \else
      \ifx\@tempb\kulemt@copyright@english
        \let\@tempb\@tempa \let\@tempa\kulemt@copyright@english \fi
      \def\@tempc{\@tempa \vskip\baselineskip}%
      \expandafter\@tempc\@tempb
    \fi
  \endgroup
  \clearpage}
%    \end{macrocode}
% \end{macro}
%
% \subsubsection{Typesetting the filing card}
% \DefinePageStyle{filingcard}
% The filing card uses its own page style \pstyle{filingcard}, typeset in the
% master language. Its ruled header contains the faculty name and the academic
% year. No footer is used.
%    \begin{macrocode}
\makepagestyle{filingcard}
\makeheadrule{filingcard}{\textwidth}{\normalrulethickness}
\makeevenhead{filingcard}{\kulemt@selectmasterlanguage
  KU~Leuven \kulemt@fac@name}{}{\kulemt@acyear@t}
\makeoddhead{filingcard}{\kulemt@selectmasterlanguage
  KU~Leuven \kulemt@fac@name}{}{\kulemt@acyear@t}
%    \end{macrocode}
%
% \begin{macro}{\kulemt@filingcard}
% The filing card is put on a separate page with its own page style, using
% the master language.
%    \begin{macrocode}
\def\kulemt@filingcard{\clearforchapter
  \thispagestyle{filingcard}%
  \begingroup
    \kulemt@selectmasterlanguage
%    \end{macrocode}
% First a centered title is printed.
%    \begin{macrocode}
    \begingroup
      \centering \Large
      \kulemt@ifdutch{Fiche masterproef}{Master thesis filing card}%
      \vskip 1em
    \endgroup
%    \end{macrocode}
% First the title, translated title, keywords, and article title are
% typeset with a medium space between them. The title and translated title
% are typeset in the main text language.
%    \begin{macrocode}
    \begingroup
      \parskip\medskipamount
      \@hangfrom{\textit{%
          Student\kulemt@ifand\@author{\kulemt@ifdutch{en}s}}{}: }%
        \@author\par
      \@hangfrom{\textit{Tit\kulemt@ifdutch{el}{le}}: }%
        {\kulemt@selecttextlanguage \@title}\par
      \ifx\kulemt@translatedtitle\@empty\else
        \@hangfrom{\textit{%
            \kulemt@ifdutch
             {\kulemt@selecttextlanguage
              \kulemt@ifdutch{Engel}{Nederland}se titel}%
             {\kulemt@selecttextlanguage
              \kulemt@ifdutch{English}{Dutch} title}}: }%
        \kulemt@translatedtitle\par
      \fi
      \noindent \textit{UDC}: \kulemt@udc\par
      \ifx\kulemt@keywords\@empty\else
        \@hangfrom{\textit{Keywords}: }\kulemt@keywords\par
      \fi
      \ifx\kulemt@article@title\@empty\else
        \@hangfrom{\textit{%
            \kulemt@ifdutch{Titel van het artikel}{Article title}}: }%
          \kulemt@article@title\par
      \fi
      \vskip\medskipamount
    \endgroup
%    \end{macrocode}
% Then comes the short abstract in the main text language.
%    \begin{macrocode}
    \noindent \textit{\kulemt@ifdutch{Korte inhoud}{Abstract}}:%
    \vskip 2\p@
    \begingroup \kulemt@selecttextlanguage
      \noindent\ignorespaces \kulemt@shortabstract
    \endgroup
%    \end{macrocode}
% The rest comes at the bottom of the page: master degree, promoter(s),
% assessors, and assistant(s). Between these items we put a small space.
% The \cs{raggedright} command must be used inside a group because it is
% incompatible with \cs{@hangfrom}.
%    \begin{macrocode}
    \vfill \parskip\smallskipamount
    \begingroup \raggedright
      \noindent \kulemt@master@text \par
    \endgroup
    \@hangfrom{\textit{\kulemt@paa@0}: }\kulemt@promotor\par
    \@hangfrom{\textit{\kulemt@paa@1}: }\kulemt@assessor\par
    \@hangfrom{\textit{\kulemt@paa@2}: }\kulemt@assistant\par
  \endgroup
  \clearpage}
%    \end{macrocode}
% \end{macro}
%
% \subsubsection{Printing the required pages}
% At the beginning of the document, the front matter starts with the front
% page (either the cover page or the title page). Next the copyright page
% is printed unless the first page was a cover page. If only the cover page
% or the front pages are printed, the document ends here.\\
% The \pkg{hyperref} package requires a unique printed page number. Since
% non-positive page numbers have no roman representation, the
% \cs{frontmatter} is only switched on after the copyright page.\\
% \changes{v1.3}{2011/05/13}{Required pages are now printed at the end of
%   \cs{document}}^^A
% We can't use \cs{AtBeginDocument} here, because some packages assume that
% no text is generated before the commands they add to this hook. An
% example is the externalization library of the package \pkg{tikz}. To
% avoid such problems, we simply append the commands to the definition of
% \cs{document}.
% \changes{v1.8}{2015/05/25}{Make sure \cs{@tempb} is not overwritten by
%   \cs{@tempa}}
%    \begin{macrocode}
\g@addto@macro\document{\kulemt@frontpage
  \ifkulemt@coverpage
    \def\@tempa{\end{document}}%
  \else
    \kulemt@copyrightpage
    \ifkulemt@frontpages
      \def\@tempa{\end{document}}%
    \else
      \let\@tempa\frontmatter
    \fi
  \fi
  \@tempa}
%    \end{macrocode}
% \begin{macro}{\kulemt@error@mm}
% The user must switch to the main matter herself and we make sure that she
% doesn't forget it. The command \cs{kulemt@error@mm} will be called at the
% end of the document.
% \changes{v1.4}{2011/06/06}{Make \cs{mainmatter*} work again}^^A
% Since \cs{mainmatter} tests for a trailing star, we can't add commands at
% the end of it.
%    \begin{macrocode}
\def\kulemt@error@mm{\kulemt@error{%
    You forgot to use \string\mainmatter}}
\expandafter\def\expandafter\mainmatter\expandafter{%
  \expandafter\let\expandafter\kulemt@error@mm\expandafter\relax
  \mainmatter}
%    \end{macrocode}
% \end{macro}
% At the end of the document, we first check if \cs{mainmatter} was used,
% unless only cover or front pages are printed. If a filing card is wanted,
% it is printed as back matter.
%    \begin{macrocode}
\AtEndDocument{%
  \ifkulemt@coverpage\else
    \ifkulemt@frontpages\else \kulemt@error@mm \fi
    \ifkulemt@filingcard
      \if@mainmatter \backmatter \fi
      \kulemt@filingcard
    \fi
  \fi}
%    \end{macrocode}
%
% \subsection{Front matter environments}
% \begin{environment}{preface}
% The |preface| environment holds the preface text. It has one optional
% argument, which holds the preface author. The default preface author is
% the value of the |author| option.  The preface is printed as a single
% page chapter.
% \begin{macro}{\kulemt@preface@}
% The command \cs{kulemt@preface@} remembers the argument of the |preface|
% environment until the end of the environment.
%    \begin{macrocode}
\newenvironment{preface}[1][\@author]%
 {\chapter\prefacename
  \def\kulemt@preface@{#1}}%
 {\par
  \ifx\kulemt@preface@\@empty\else
    \bigskip \raggedleft \itshape \kulemt@preface@
  \fi
  \vfill \clearpage}
%    \end{macrocode}
% \end{macro}
% \end{environment}
%
% \begin{environment}{abstract}
% The |abstract| environment is redefined as an ordinary chapter.
%    \begin{macrocode}
\renewenvironment{abstract}%
 {\chapter\abstractname}%
 {\clearpage}
%    \end{macrocode}
% \end{environment}
%
% \begin{environment}{abstract*}
% \changes{v1.6}{2012/05/13}{New optional language argument}
% The |abstract*| environment works like the |abstract| environment, but it
% uses the language from its optional argument. By default this is the
% master language.
%    \begin{macrocode}
\newenvironment{abstract*}[1][\kulemt@master@language]%
 {\expandafter\selectlanguage\expandafter{#1}%
  \abstract}%
 {\endabstract}
%    \end{macrocode}
% \end{environment}
%
% \iffalse
%</class>
% \fi
% \section{The Faculty of Engineering Science configuration file}
% \label{sec:cfgfile}
% \iffalse
%<*config>
% \fi
% \def\generalname{kulemt.cfg}^^A For the \changes command
% \subsection{Definition of the masters}
% Note: To reuse this information in the manual, it must be the first
% section in the configuration file.
% ^^A \changes{v0.99}{2010/02/21}{Colors \texttt{wit} changed}
% ^^A \changes{v0.99}{2010/02/21}{Contact info \texttt{arc} added}
% ^^A \changes{v0.99}{2010/02/27}{Info modified for \texttt{arc}, \texttt{cit},
% ^^A   \texttt{cws}, \texttt{wit}, \texttt{wtk}, \texttt{ecit}, \texttt{mai}}
% \changes{v1.4}{2011/06/07}{New master \texttt{mse} replaces \texttt{mvt}}
% \changes{v1.6}{2012/05/13}{All master titles and options updated to the
%   2012 situation}
% \changes{v1.7}{2013/04/12}{2013 master titles and options}
% \changes{v1.7}{2013/05/01}{Disallow master options for some masters}
% \changes{v1.8a}{2015/06/01}{Update Nano options and add some English masters}
%    \begin{macrocode}
%% This kulemt.cfg file holds all master dependent information for
%% the KU Leuven engineering master thesis class.
%% Author: Luc Van Eycken (Luc.VanEycken@esat.kuleuven.be)
%% If you modify this file:
%% * provide feedback to the original author
%% * please adjust the date [YYYY/MM/DD]
\ProvidesFile{kulemt.cfg}[2013/05/01]
%% Define known masters and their options
%%   The definition of the master contains the following elements:
%%    1. "N" or "E" : the language of the master
%%                    "N" for dutch, "E" for English
%%    2. Number for faculty identification (use braces if > 1 digit)
%%       0 = multiple faculties
%%       1 = Faculty of Engineering Science
%%    3. "F" or "N" : if "F", a filing card is always required
%%    4. Master colors "{bg:fg}" or "{bg}", with each color a comma
%%       separated list of C,M,Y,K fractions.
%%    5. Master title between braces
%%    6. Optional copyright contact info {<address>:<phone>:<email>}
%%       Use faculty information if empty
%%    7. Optional list of master option abbreviations
%%       Each option is surrounded by braces and consists of an
%%       abbreviation, followed by ":" and the title of the option.
%%       Optionally the list can start with a list of abbreviations
%%       between square brackets. If this list is not empty, an error
%%       is raised when no master option is specified by the student.
%%       If the list equals "-", no master options are allowed.
%%    8. Optional list of obsolete master option abbreviations.
%%       The list has the same format as the list of master options.
%%       You have to make sure that the abbreviations don't conflict
%%       with those of the master options. The convention is to append
%%       a dot and the last year it was valid.
%%
\kulemt@div@master{Dutch initial masters}
\kulemt@def@master{arc}{N1N{0.93,0.52,0.35,0.11:0,0,0,0}%
  {Master of Science in de ingenieurswetenschappen: architectuur}{}{[-]}}
\kulemt@def@master{bin}{N0N{}%
  {Master of Science in de bio-informatica}}
\kulemt@def@master{bmt}{N1N{0.6,0,0.3,0}%
  {Master of Science in de ingenieurswetenschappen:
   biomedische~technologie}}
\kulemt@def@master{bwk}{N1N{0.2,0.7,1,0:0,0,0,0}%
  {Master of Science in de ingenieurswetenschappen: bouwkunde}{}%
  {{ct:optie Civiele techniek}%
   {gt:optie Gebouwentechniek}%
   {vk:optie Verkeerskunde}}}
\kulemt@def@master{cit}{N1N{0.9,0.26,1,0.13:0,0,0,0}%
  {Master of Science in de ingenieurswetenschappen:
   chemische~technologie}{}%
  {{cbpe:optie Chemische en biochemische proces engineering}%
   {me:optie Milieu engineering}%
   {pe:optie Product engineering}}%
  {{cbr.2012:optie Chemische en biochemische reactorkunde}%
   {ct.2012:optie Chemische technologie}%
   {mv.2012:optie Milieu en veiligheid}}}
\kulemt@def@master{cws}{N1F{0,0,1,0}%
  {Master of Science in de ingenieurswetenschappen:
   computerwetenschappen}%
  {\kulemt@ifdutch{het}{the} Departement Computerwetenschappen,
   Celestijnenlaan 200A bus 2402, B-3001 Heverlee:%
   +32-16-327700:info@cs.kuleuven.be}%
  {{ai:hoofdspecialisatie Artifici\"ele intelligentie}%
   {ci:hoofdspecialisatie Computationele informatica}%
   {db:hoofdspecialisatie Databases}%
   {gs:hoofdspecialisatie Gedistribueerde systemen}%
   {mmc:hoofdspecialisatie Mens-machine communicatie}%
   {se:hoofdspecialisatie Software engineering}%
   {vs:hoofdspecialisatie Veilige software}}%
  {{ai.2011:optie Artifici\"ele intelligentie}%
   {gs.2011:optie Gedistribueerde systemen}%
   {mmc.2011:optie Mens-machine communicatie}%
   {vs.2011:optie Veilige software}}}
\kulemt@def@master{elt}{N1N{0,0.2,0.7,0}%
  {Master of Science in de ingenieurswetenschappen: elektrotechniek}%
  {ESAT, Kasteelpark Arenberg 10 postbus 2440,
   B-3001 Heverlee:+32-16-321130:info@esat.kuleuven.be}%
  {[eg,im]%
   {eg:optie Elektronica en ge\"{\i}ntegreerde schakelingen}%
   {im:optie Ingebedde systemen en multimedia}}%
  {{ge.2012:optie Ge\"{\i}ntegreerde elektronica}%
   {ms.2012:optie Multimedia en signaalverwerking}%
   {tt.2012:optie Telecommunicatie en telematica}}}
\kulemt@def@master{ene}{N1N{0.5,0,1,0}%
  {Master of Science in de ingenieurswetenschappen: energie}{}{[-]}}
\kulemt@obsolete@master{gmk}{N1N{0.8,0.6,0,0:0,0,0,0}%
  {Master of Science in de ingenieurswetenschappen:
   geotechniek en mijnbouwkunde}}
\kulemt@def@master{mtk}{N1N{0.3,0,0.3,0}%
  {Master of Science in de ingenieurswetenschappen: materiaalkunde}{}%
  {{mb:optie Materialen in de biomedische sector}%
   {mk:optie Metalen en keramieken}%
   {mn:optie Materialen voor nanotechnologie}%
   {pc:optie Polymeren en composieten}%
   {pp:optie Productie en processen}}}
\kulemt@def@master{vlit}{N1N{0,0,0.33,0}%
  {Master of Science in de ingenieurswetenschappen:
   verkeer, logistiek en intelligente transportsystemen}%
  {Centre for Industrial Management, Celestijnenlaan 300A Bus 2422,
   B-3001 Heverlee:+32-16-322567}%
  {{lt:optie Logistiek en transport}%
   {vi:optie Verkeer en Infrastructuur}}}
\kulemt@obsolete@master{mtw}{N0N{}%
  {Master in de milieutechnologie en de milieuwetenschappen}}
\kulemt@def@master{nan}{N1N{0,0.8,0.7,0:0,0,0,0}%
  {Master of Science in de nanowetenschappen en de nanotechnologie}{}%
  {{nm:optie Nanomaterialen en nanochemie}%
   {ne:optie Nano-elektronicaontwerp}%
   {nc:optie Nanocomponenten en nanofysica}%
   {nb:optie Nanobiotechnologie}}%
  {{bi.2014:afstudeerrichting bio-ingenieur}%
   {ir.2014:afstudeerrichting burgerlijk ingenieur}%
   {nw.2014:afstudeerrichting natuurwetenschappen}}}
\kulemt@def@master{sta}{N0N{}%
  {Master of Science in de Statistiek}{}%
  {{asm:specialisatie Algemene statistische methodologie}%
   {bm:specialisatie Biometrie}%
   {bs:specialisatie Business statistiek}%
   {is:specialisatie Industri\"ele statistiek}%
   {sgp:specialisatie Statistiek in de sociale, gedrags- en
        pedagogische wetenschappen}%
   {so:specialisatie Statistiek en onderwijs}}}
\kulemt@def@master{wit}{N1F{0.9,0.94,0.02,0.07:0,0,0,0}%
  {Master of Science in de ingenieurswetenschappen:
   wiskundige~ingenieurstechnieken}%
  {\kulemt@ifdutch{het}{the} Departement Computerwetenschappen,
   Celestijnenlaan 200A bus 2402, B-3001 Heverlee:%
   +32-16-327700:info@cs.kuleuven.be}}
\kulemt@def@master{wtk}{N1N{0.6,0.3,0,0:0,0,0,0}%
  {Master of Science in de ingenieurswetenschappen: werktuigkunde}{}{[-]}}
%
\kulemt@div@master{English initial masters}
\kulemt@def@master{ebmt}{E1N{0.6,0,0.3,0}%
  {Master of Science in Biomedical~Engineering}}
\kulemt@def@master{ebin}{E0N{}%
  {Master of Science in Bioinformatics}}
\kulemt@def@master{ecit}{E1N{0.9,0.26,1,0.13:0,0,0,0}%
  {Master of Science in Chemical~Engineering}{}%
  {{cbpe:option Chemical and biochemical process engineering}%
   {me:option Environmental engineering}%
   {pe:option Product engineering}}}
\kulemt@def@master{ect}{E1N{0.9,0.26,1,0.13:0,0,0,0}%
  {Master of Science in Chemical Engineering (Engineering Rheology)}}
\kulemt@def@master{ecws}{E1F{0,0,1,0}%
  {Master of Science in Engineering: Computer Science}%
  {\kulemt@ifdutch{het}{the} Departement Computerwetenschappen,
   Celestijnenlaan 200A bus 2402, B-3001 Heverlee:%
   +32-16-327700:info@cs.kuleuven.be}%
  {{ai:specialisation Artificial Intelligence}%
   {ss:specialisation Secure Software}}}
\kulemt@def@master{eelt}{E1N{0,0.2,0.7,0}%
  {Master of Science in Electrical~Engineering}%
  {Departement Elektrotechniek, Kasteelpark Arenberg 10 postbus 2440,
    B-3001 Heverlee:+32-16-321130:info@esat.kuleuven.be}%
  {[ei,em]%
   {ei:option Electronics and Integrated Circuits}%
   {em:option Embedded Systems and Multimedia}}}
\kulemt@def@master{eene}{E1N{0.5,0,1,0}%
  {Master of Science in Engineering: Energy}{}{[-]}}
\kulemt@def@master{ekene}{E1N{0.5,0,1,0}%
  {EIT-KIC Master in Energy}{}{[-]}}
\kulemt@def@master{ememn}{E1N{0.5,0,1,0}%
  {Erasmus Mundus Joint Master of Economics and
   Management of Network~Industries}}
\kulemt@def@master{emtk}{E1N{0.3,0,0.3,0}%
  {Master of Science in Materials Engineering}{}%
  {{mc:option Metals and Ceramics}%
   {mn:option Materials for Nanotechnology}%
   {pc:option Polymers and Composites}}}
\kulemt@def@master{enan}{E1N{0,0.8,0.7,0:0,0,0,0}%
  {Master of Science in Nanoscience and Nanotechnology}{}%
  {{nm:option Nanomaterials and Nanochemistry}%
   {ne:option Nanoelectronic Design}%
   {nd:option Nanodevices and Nanophysics}%
   {nb:option Nanobiotechnology}}%
  {{be.2014:major subject Bioscience engineering}%
   {eng.2014:major subject Engineering}%
   {ns.2014:major subject Natural sciences}}}
\kulemt@def@master{emnan}{E0N{0,0.8,0.7,0:0,0,0,0}%
  {Erasmus Mundus Master of Science in
   Nanoscience and Nanotechnology}{}%
  {{bb:graduation option Biophysics and Bionanotechnology}%
   {ne:graduation option Nanoelectronics}%
   {nn:graduation option Nanophysics and Nanochemistry}}}
\kulemt@def@master{esta}{E0N{}%
  {Master of Science in Statistics}{}%
  {{ars:option All Round Statistics}%
   {bm:option Biometrics}%
   {bs:option Business Statistics}%
   {gsm:option General Statistical Methodology}%
   {is:option Industrial Statistics}%
   {qas:abridged programme --
        Quantitative Analysis in the Social Sciences}%
   {sbe:option Social, Behavioral and Educational Statistics}}}
\kulemt@def@master{ewit}{E1F{0.9,0.94,0.02,0.07:0,0,0,0}%
  {Master of Science in Mathematical Engineering}%
  {\kulemt@ifdutch{het}{the} Departement Computerwetenschappen,
   Celestijnenlaan 200A bus 2402, B-3001 Heverlee:%
   +32-16-327700:info@cs.kuleuven.be}}
\kulemt@def@master{ewtk}{E1N{0.6,0.3,0,0:0,0,0,0}%
  {Master of Science in Mechanical Engineering}{}{[-]}}
%
\kulemt@div@master{Post-initial masters}
\kulemt@def@master{cms}{E1N{}%
  {Master of Science in Conservation of Monuments and Sites}}
\kulemt@def@master{mai}{E0N{}%
  {Master of Science in Artificial Intelligence}%
  {\kulemt@ifdutch{het}{the} Departement Computerwetenschappen,
   Celestijnenlaan 200A bus 2402, B-3001 Heverlee:%
   +32-16-327700:info@cs.kuleuven.be}%
  {{cs:option Cognitive Science}%
   {ecs:option Engineering and Computer Science}%
   {slt:option Speech and Language Technology}}}
\kulemt@def@master{mhs}{E1N{}%
  {Master of Science in Human Settlements}}
\kulemt@obsolete@master{mim}{E1N{}%
  {Master of Industrial Management}{}%
  {{ese:option Environment, Safety and Energy}%
   {ict:option Information and Communication Technology}%
   {plp:option Production and Logistics Planning}}}
\kulemt@def@master{mms}{N0N{}%
  {Master of Science in de medische stralingsfysica}}
\kulemt@def@master{mne}{E1N{}%
  {Master of Science in Nuclear Engineering}}
\kulemt@def@master{mse}{E1N{}%
  {Master of Science in Safety Engineering}{}
  {[p,ps]%
   {p:option Prevention}%
   {ps:option Process Safety}}}
\kulemt@def@master{mss}{E0N{}%
  {Master of Science in Space Studies}{}%
  {{slpbm:major subject: Space Law, Policy, Business and Management}%
   {ss:major subject: Space Sciences}%
   {sta:major Subject: Space Technology and Applications}}}
\kulemt@obsolete@master{mvt}{N1N{}%
  {Master in de veiligheidstechniek}}
\kulemt@def@master{usp}{E1N{}%
  {Master of Science in Urbanism and Strategic Planning}{}%
  {{sp:option Spatial Planning}%
   {u:option Urbanism}}}
%
\kulemt@end@master@def
%    \end{macrocode}
%
% \subsection{Local definitions}
% If you don't agree with the default titles of the jury members (promoter,
% assessor, assistant), you can redefine them here. These definitions are
% inspired by the official KU~Leuven translations and by suggestions from
% the faculty.
%    \begin{macrocode}
\def\kulemt@paa@#1{%
  \ifcase #1%
    \kulemt@ifdutch
      {Promotor\kulemt@ifand\kulemt@promotor{en}{}}%
      {Thesis supervisor\kulemt@ifand\kulemt@promotor{s}{}}%
  \or
    \kulemt@ifdutch
      {Assessor\kulemt@ifand\kulemt@assessor{en}{}}%
      {Assessor\kulemt@ifand\kulemt@assessor{s}{}}%
  \or
    \kulemt@ifdutch
      {Begeleider\kulemt@ifand\kulemt@assistant{s}{}}%
      {Mentor\kulemt@ifand\kulemt@assistant{s}{}}%
  \fi}
%    \end{macrocode}
% \iffalse
%</config>
% \fi
%
%
% \Finale
\endinput

%% Local Variables:
%% ispell-check-comments: exclusive
%% End:
